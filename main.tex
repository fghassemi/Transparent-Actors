\documentclass{llncs}%[a4paper,UKenglish,cleveref, autoref, thm-restate]{lipics-v2019}
%This is a template for producing LIPIcs articles. 
%See lipics-manual.pdf for further information.
%for A4 paper format use option "a4paper", for US-letter use option "letterpaper"
%for british hyphenation rules use option "UKenglish", for american hyphenation rules use option "USenglish"
%for section-numbered lemmas etc., use "numberwithinsect"
%for enabling cleveref support, use "cleveref"
%for enabling autoref support, use "autoref"
%for anonymousing the authors (e.g. for double-blind review), add "anonymous"
%for enabling thm-restate support, use "thm-restate"

%\graphicspath{{./graphics/}}%helpful if your graphic files are in another directory

\bibliographystyle{plainurl}% the mandatory bibstyle

%\usepackage[usenames,dvipsnames]{color}
%\usepackage[usenames,dvipsnames]{xcolor}
\usepackage{listings}





%-----------------------------------------
%------------------------
\lstdefinelanguage{Hrebeca}{
	morekeywords={softwareclass, physicalclass,changemode, mode,inv,msgsrv,guard,delay,statevars, int,float,real,knownrebecs,if,else,self,main,Wire,CAN,bool,reactiveclass,true,false,guard},
	otherkeywords={=>,<-,<\%,<:,>:,\#,@,'},
	sensitive=true,
	morecomment=[l]{//},
	morecomment=[n]{/*}{*/},
	morestring=[b]",
	morestring=[b]',
	morestring=[b]"""
}
\lstset{
	language=Hrebeca,
	aboveskip=3mm,
	belowskip=3mm,
	showstringspaces=false,
	columns=flexible,
	xleftmargin=10mm,
	basicstyle={\scriptsize},
	keywordstyle=\color{blue},
	numbers=left,
	numberstyle=\color{black},
	numbersep=7pt,
	stepnumber=1,
	breaklines=true,
	breakatwhitespace=true,
	tabsize=3,
	numberblanklines=false,
	frame=l
}
%--------
%----------LISTING --------------
\definecolor{codegreen}{rgb}{0,0.6,0}
\definecolor{codegray}{rgb}{0.5,0.5,0.5}
\definecolor{codepurple}{rgb}{0.58,0,0.82}
\definecolor{backcolour}{rgb}{0.95,0.95,0.92}

\lstdefinestyle{mystyle}{
    backgroundcolor=\color{backcolour},   
    commentstyle=\color{codegreen},
    keywordstyle=\color{blue},
    numberstyle=\tiny\color{codegray},
    stringstyle=\color{codepurple},
    basicstyle=\ttfamily\footnotesize,
    breakatwhitespace=false,         
    breaklines=true,                 
    captionpos=b,                    
    keepspaces=true,                 
    numbers=left,                    
    numbersep=5pt,                  
    showspaces=false,                
    showstringspaces=false,
    showtabs=false,                  
    tabsize=2
}


\lstdefinelanguage{LF}{
  keywords={after, reaction, preamble, target, reactor, composite, trigger, input, output, new, action, physical, clock, actor, int, main, Main, state, timer, init, realtime, msec},
  identifierstyle=\color{black},
  sensitive=false,
  comment=[l]{//},
  morecomment=[s]{/*}{*/},
  morestring=[b]',
  morestring=[b]"
}

\lstdefinestyle{lfstyle}{
  language=LF,
  basicstyle=\ttfamily\footnotesize,
  backgroundcolor=\color{backcolour},   
  keywordstyle=\bfseries,
  breakatwhitespace=false,     
  breaklines=true,         
  captionpos=b,          
  keepspaces=true,         
  numbers=left,          
  numbersep=5pt,
  numberstyle=\tiny,
  showspaces=false,        
  showstringspaces=false,
  showtabs=false,          
  tabsize=1,
  xleftmargin=5mm,
  xrightmargin=5mm
}
\lstdefinelanguage{ABS}{
  keywords={interface, class, implements, Fut, Int, Unit},
  identifierstyle=\color{black},
  sensitive=false,
  comment=[l]{//},
  morecomment=[s]{/*}{*/},
  morestring=[b]',
  morestring=[b]"
}

\lstdefinestyle{absstyle}{
  language=ABS,
  basicstyle=\ttfamily\footnotesize,
  backgroundcolor=\color{backcolour},   
  keywordstyle=\bfseries,
  breakatwhitespace=false,     
  breaklines=true,         
  captionpos=b,          
  keepspaces=true,         
  numbers=left,          
  numbersep=5pt,
  numberstyle=\tiny,
  showspaces=false,        
  showstringspaces=false,
  showtabs=false,          
  tabsize=1,
  xleftmargin=5mm,
  xrightmargin=5mm
}
\lstdefinelanguage{ADT}{
  keywords={sort,var,eqn,cons,map,if},
  identifierstyle=\color{black},
  sensitive=false,
  comment=[l]{//},
  morecomment=[s]{/*}{*/},
  morestring=[b]',
  morestring=[b]"
}

\lstdefinestyle{adtstyle}{
  language=ADT,
  basicstyle=\ttfamily\footnotesize,
  backgroundcolor=\color{backcolour},   
  keywordstyle=\bfseries,
  breakatwhitespace=false,     
  breaklines=true,         
  captionpos=b,          
  keepspaces=true,         
  numbers=left,          
  numbersep=5pt,
  numberstyle=\tiny,
  showspaces=false,        
  showstringspaces=false,
  showtabs=false,          
  tabsize=1,
  xleftmargin=5mm,
  xrightmargin=5mm
}

\lstset{language=ABS,style=absstyle}
\lstset{language=ADT,style=adtstyle}
\lstset{language=LF,style=lfstyle}
\lstset{style=mystyle}


\usepackage{fancybox}
\usepackage[utf8]{inputenc}
\usepackage{amssymb}
\usepackage{latexsym}
%\setcounter{tocdepth}{3}
\setcounter{secnumdepth}{5}
\usepackage{graphicx}
\usepackage{alltt}
\usepackage{multicol}
\usepackage{amsmath}
\usepackage{amsfonts}
\usepackage{mathtools}
%\usepackage{ragged2e}
\usepackage{mathpartir}
\usepackage{syntax}
\usepackage{comment}
%\usepackage{ifthen}
\usepackage{booktabs}
\usepackage{multirow}
%\usepackage{biblatex}
%\addbibresource{ref.bib}
\usepackage{wasysym}
\usepackage{pifont}% http://ctan.org/pkg/pifont
\usepackage{tikz}
\usetikzlibrary{arrows,shapes,positioning,decorations.pathmorphing,shapes,calc}
%-----COMMANDS----------------------
\newcommand{\cmark}{\ding{51}}%
\newcommand{\xmark}{\ding{55}}

\newcounter{sarrow}
\newcommand\xrsquigarrow[1]{%
\stepcounter{sarrow}%
\mathrel{\begin{tikzpicture}[baseline= {( $ (current bounding box.south) + (0,-0.5ex) $ )}]
\node[inner sep=.5ex] (\thesarrow) {$\scriptstyle #1$};
\path[draw,<-,decorate,
  decoration={zigzag,amplitude=0.7pt,segment length=1.2mm,pre=lineto,pre length=4pt}] 
    (\thesarrow.south east) -- (\thesarrow.south west);
\end{tikzpicture}}%
}

\providecommand\longrightarrowRHD{\relbar\joinrel\relbar\joinrel\mathrel\RHD}
\providecommand\longrightarrowrhd{\relbar\joinrel\relbar\joinrel\mathrel\rhd}
\makeatletter
\providecommand*\xrightarrowRHD[2][]{\ext@arrow 0055{\arrowfill@\relbar\relbar\longrightarrowRHD}{#1}{#2}}
\providecommand*\xrightarrowrhd[2][]{\ext@arrow 0055{\arrowfill@\relbar\relbar\longrightarrowrhd}{#1}{#2}}
\makeatletter



\newcommand{\Rule}[2]{                                  % operational rule
	\frac{\raisebox{.7ex}{\normalsize{$#1$}}}
	{\raisebox{-1.0ex}{\normalsize{$#2$}}}}

\newcommand{\FAxiom}[1]{                                  % operational rule
	{\normalsize {#1}}
}
\newcommand{\overto}[1]{\stackrel{#1}{%
		\overrightarrow{\smash{\,\scriptsize{\phantom{#1}}\,}}}}
\newcommand{\aoverto}[1]{\xrightarrowrhd{#1}}
\newcommand{\noverto}[1]{\xrightarrowRHD{#1}}
\newcommand{\stoverto}[1]{\stackrel{#1}{%
		\lhook\joinrel\xrightarrow{\smash{\,\scriptsize{\phantom{#1}}\,}}}}
\newcommand{\boverto}[1]{\xrsquigarrow{#1}}

\newcommand{\MS}[1]{{\color{blue}\textbf{MS:} #1}}
\newcommand{\FG}[1]{{\color{green}\textbf{FG:} #1}}
\newcommand{\HH}[1]{{\color{green}\textbf{HH:} #1}}
\newcommand{\fixme}[1]{{\color{red}\textbf{fixme:} #1}}


%------DEFINITIONS------------------------

\def\take{\mathit{take}}
\def\select{\mathit{select}}
\def\put{\mathit{put}}
\def\snd{\mathit{snd}}
\def\remove{\mathit{remove}}
\def\insert{\mathit{insert}}
\def\Act{\mathit{Act}}
\def\Mode{\mathit{Mode}}
\def\inv{\mathit{inv}}
\def\guard{\mathit{guard}}
\def\trigger{\mathit{trigger}}
\def\flow{\mathit{flow}}

\def\size{\mathit{size}}
\def\empt{\mathit{empty}}
\def\send{\mathit{send}}
\def\receive{\mathit{receive}}
\def\transfer{\mathit{transfer}}
\def\highestPriority{\mathit{highest\_priority}}
\def\acquireTimer{\mathit{acquire\_timer}}
\def\MName{\mathit{Name}}
\def\MoName{\mathit{ModeName}}
\def\Content{\mathit{Content}}
\def\Msg{\mathit{Msg}}
\def\rcv{\mathit{rcv}}
\def\sndr{\mathit{sndr}}
\def\name{\mathit{name}}
\def\priority{\mathit{priority}}
\def\Config{\mathit{Config}}
\def\delay{\mathit{delay}}
\def\ID{\mathit{ID}}
\def\set{\mathit{Set}}
\def\Queue{\mathit{Queue}}
\def\Fifo{\mathit{Fifo}}
\def\Var{\mathit{Var}}
\def\Value{\mathit{Value}}
\def\Stmt{\mathit{Stmt}}
\def\ODE{\mathit{ODE}}
\def\hdlrs{\mathit{hdlrs}}
\def\handle{\mathit{handle}}
\def\mdelay{\mathit{MsgDelay}}
\def\Neibour{\mathit{Neibour}}
\def\Media{\mathit{AbsNetwork}}
\def\false{\mathit{false}}
\def\true{\mathit{true}}
\def\error{\mathit{error}}
\def\local{\mathit{LocalSt}}
\def\gstate{\mathit{GlobalSt}}

\def\tag{\mathit{tg}}
\def\TagT{\mathit{Tag}}
\def\env{\mathit{env}}
\def\envT{\mathit{Env}}
\def\cond{\mathit{cond}}
\def\now{\mathit{now}}
\def\rt{\mathit{rt}}
\def\mode{\mathit{mode}}
\def\info{\mathit{info}}
\def\ar{\mathit{ar}}
\def\dt{\mathit{dt}}
\def\msgaPart{\mathit{msg}}
\def\body{\mathit{body}}
\def\int{\mathbb{N}}
\def\buffer{\mathit{Buffer}}
\def\bag{\mathit{Bag}}
\def\Nat{\mathbb{N}}
\def\msgHandlr{\mathit{Handlr}}
\def\aqu{\mathit{aqu}}
\def\tail{\mathit{tail}}
\def\head{\mathit{head}}
\def\domain{\mathit{domain}}
\def\self{\mathit{self}}
\def\eval{\mathit{eval}}
\def\expr{\mathit{expr}}
\def\Expr{\mathit{Expr}}
\def\include{\mathit{include}}
\def\bool{\mathit{bool}}
\def\cur{\mathit{cur}}
\def\true{\mathit{true}}
\newtheorem{defn}{Definition}


\let\emptyset\varnothing
%---------------------
%----------------------------------
\title{Transparent Actor Models}
\author{Fatemeh, Ehsan, Hossein, Marjan}{Dummy University Computing Laboratory, [optional: Address], Country \and My second affiliation, Country \and \url{http://www.myhomepage.edu} }{johnqpublic@dummyuni.org}{https://orcid.org/0000-0002-1825-0097}{(Optional) author-specific funding acknowledgements}
\authorrunning{J.\,Q. Public and J.\,R. Public} %TODO mandatory. First: Use abbreviated first/middle names. Second (only in severe cases): Use first author plus 'et al.'

\Copyright{Fatemeh, Ehsan, Hossein, Marjan} %TODO mandatory, please use full first names. LIPIcs license is "CC-BY";  http://creativecommons.org/licenses/by/3.0/

\ccsdesc[100]{\textcolor{red}{Replace ccsdesc macro with valid one}} %TODO mandatory: Please choose ACM 2012 classifications from https://dl.acm.org/ccs/ccs_flat.cfm 

\keywords{Dummy keyword} %TODO mandatory; please add comma-separated list of keywords

\begin{document}



% a short form should be given in case it is too long for the running head
%\titlerunning{Choreography-based Runtime Verification}


%\author{XX \and YY \and ZZ}
%\institute{M. Samadi \at
%	University of Tehran, Iran \\
%	\email{mbh.samadi@ut.ac.ir}
%	\and
%	F. Ghassemi \at
%	University of Tehran, Iran \\
%	\email{fghassemi@ut.ac.ir} \and
%	R. Khosravi \at
%	University of Tehran, Iran \\
%	\email{r.khosravi@ut.ac.ir}    }

\maketitle


%---------------------------------------------

\begin{abstract}
The classical computational model of actors has been proposed for modeling event-based asynchronous communicating distributed systems. This model abstracts away from the effect of network by defining how messages are dispatched among the entities.  Message delivery is guaranteed with no restriction on the order of messages. Furthermore, no policy is considered for handling the received messages. Many different formal modeling languages have been proposed based on this computational model. Each language imposes a set of assumptions on the delivery of messages or the policy by which the buffer messages are handled. These assumptions may depend on the domain of systems and the network settings. These assumptions are implicitly considered in the semantics of languages, which makes the comprehension of languages cumbersome. Following the same computational model of actors, we propose a generic framework, called transparent actor model, in which such assumptions become explicit in the semantics of languages through implementing a set of interfaces. To illustrate the applicability of our framework, we formalize the semantics of various actor-based languages in the literature in terms of our framework. This unified framework alleviates to generalize the existing theories for different settings.

%%The classical computational model of Actors for asynchronous communicating distributed systems considers a very abstract model of network by which only packet delivery is guaranteed. The policy by which the message buffers are handled is not defined. Different network settings like Ethernet, wireless, CAN bus have different effects on how messages are dispatched between entities in a distributed systems. Different policies on the delivery of messages, e.g., overwriting old messages, dropping when the buffer is overflow, etc, or taking messages from the buffer regarding the priority or timestamps of messages have been ignored in this classical computational model. For the verification of distributed systems with model checking technique, different aspects of systems like timing, probabilistic, etc are considered to resolve the non-determinism. In practice, this abstract network model is boiled down to one specific model with concrete network model, policies, and system aspects.  %, e.g., to bag or FIFO queues for out-of-order and in-order delivery semantics.  

%The behavior of network protocols and different policies for taking messages from the actor buffer have been weaved into the semantics of actors and this makes revisiting the actor-based framework like Rebeca for a new network settings. In addition to message transmission, there are variations at how the buffer of actors are managed to consider over-flow or the priority of messages for processing. 

%%We aim at providing a generic framework for actor models %To consider these semantic variations, we factor out a network model and buffer management model from the actor model 
%% a set of interface functions by which the modeler can define its desired network setting and buffer management policies. %semantics for the four phases of a point-to-point message communication: \textit{sending}, \textit{transmission}, \textit{receiving} , and \textit{taking}. 
%%We define a set of abstract data types for the parameters of these functions and global states of the Actor model to encapsulate the aspects of systems.

\end{abstract}


\section{Introduction}

Sematic Specification Framework for actor-based languages

Framework:
constant: Reuse the framework ...
inherit the rules of Timed Rebeca ... ABS 

Variable: policies - arguments ...
(Future and things that we did not consider)

When designing a new language: points of decision 

proving different theorems

The computational model of actors \cite{Hewitt77,Agha90} has been introduced to model concurrent and distributed systems. Such modeling has become very popular in practice. Scala programming language supports actor-models. Actors, the primitives of computation, are independent, well encapsulated, and of course, run concurrently. Each actor has its own variables, and its encapsulation prohibits other actors from accessing its variables directly. Each actor communicates with others only through message passing and owns a mailbox with a unique address to store the received messages. An actor's behavior is defined in terms of a set of message handlers that specify how the actor reacts upon processing each received message. In this model, message delivery is guaranteed but is not in-order; if an actor sends two messages to a destination, they will be received in any order. This assumption implicitly abstracts away from the effects of the network, i.e., delays over different routing paths, message conflicts, etc., and consequently makes it a suitable modeling framework for concurrent and distributed applications. 


Many different formal modeling languages, like ABS \cite{ABS}, Timed Rebeca \cite{SirjaniK16}, Lingua Franca \cite{DBLP:conf/cyphy/LohstrohRGDCLS19}, have been proposed based on this computational model (see also \cite{survay}). Each language considers a set of policies on the delivery/processing of messages or their maintenance in the buffers. These policies may depend on the domain of systems and network settings. Different network settings like Ethernet, wireless, CAN bus, etc. have different effects on how messages are dispatched among entities in a distributed system. Different policies on the receipt of messages, e.g., overwriting old messages, dropping when the buffer is overflow, etc., or taking messages from the buffer based on the priority or timestamps of messages have been considered in these languages. These policies are not explicitly explained in the definition of languages and their effects can be only followed up in the semantic rules of languages. However, one can not easily understand from the semantic rules the set of policies considered in the language. %leading to those assumptions of semantic rules. 


%For the verification of distributed systems with the model checking technique, different aspects of systems like configuration of actors, timing or probabilistic aspects, and coordination are considered to define the behavior of the systems. The rules focusing on these aspects are interweaved by those assumptions on the networks and policies of message handlers. 
To have a precise comprehension of the policies of an existing language or to facilitate designing a new language, we provide a semantic specification framework for actor-based languages by which such policies are %each separate concern is 
explicitly identified. Our framework provides a set of template rules that their assumptions may need to be adjusted with regard to the policies and network settings. We explicitly identify those assumptions that show policies. Each policy explicitly reveals a decision point in realization of the actor model leading to variation of languages. We have defined the rules based on the interaction between two actors and network to realize a point-to-point communication between actors. To capture the policies at the buffer level, we also consider the interaction between actors/network and their buffers. We consider three phases for a point-to-point message communication as illustrated in Figure \ref{Fig::schema}: 
\begin{itemize}
\item\textit{Sending}: Actors push their messages into the buffer of the abstract network entity. %The properties of the network settings defines how the message is added to the local state of the network. %At this phase the reliability is considered. 

\item\textit{Transferring}: The abstract network entity delivers messages from its buffer to the receiver. %In this phase, the affect of network setting is considered as reliable/unreliable delivery, the delay of messages, variations at message delivery such as  in-order, out-of-order, or causal-order delivery, scheduling of messages arrived at the same time, etc.

%\item\textit{Receiving}: The abstract network entity pushes the messages from the network buffer into the buffer of receivers. %The policies of the receiver on handling buffer overflow, overwriting old messages or dropping redundant messages are considered at this phases. 

\item\textit{Taking}: The actor chooses a message from the buffer to process it. %Policies at this phase realizes the buffer management concerns from the point-view of the application. It may consider the time-stamp or the priority of messages to defined which message should be processed at first.  
	
\end{itemize}

\begin{figure}[htbp]
	\centering
	\usetikzlibrary{shapes,positioning,decorations.pathmorphing}

\begin{tikzpicture}%[arrow/.style = {draw=#1,-{Stealth[]}, shorten >=1mm, shorten <=1mm}]

\draw  (0,2.5) ellipse (1 and 1.5);
\draw  (-0.6,2.1) rectangle (0.6,1.4);
\node at (0,1.7) {$\it Buffer$};

\draw  (4,2.5) ellipse (1 and 1.5);
\draw  (3.4,2.1) rectangle (4.6,1.4);
\node at (4,1.7) {$\it Buffer$};

\draw[<->,dashed] (-0.3,2.9) -- (4.5,2.9) ;
\node at (2,3.1) {$\it msg$};

 \node[cloud, draw, align=left, cloud puffs=15,cloud puff arc=110, aspect=3, inner sep=1mm] at (2,0) {Abstract Network\\ \\};
 \draw  (1.4,0) rectangle (2.6,-.7);
\node at (2,-0.4) {$\it Buffer$};
 \draw[->] (4.5,2.9) to [out=-35, in=0] (2.6,-.1);%(4.6,0.4);
% \draw[->] (4.7,0.4) to [out=200, in=-20] (-0.7,0.3);
 \draw[->] (1.4,-0.1) to [out=180, in=225] (-0.6,1.4);
 \draw[->] (-0.6,2.1) to [out=130, in=225] (-0.3,2.9);
 
 \node[shape=circle,draw=black,outer sep =0, inner sep=1] at (5.4,1.5) {\small$1$};
 \node at (6.2,1.5) {${\it sending}$};
% \node[shape=circle,draw=black,outer sep =0, inner sep=1] at (1.4,-0.4) {\small$2$};
 %\node at (2.3,-0.4) {${\it transfer}$};
  \node[shape=circle,draw=black,outer sep =0, inner sep=1] at (-3,0.8) {\small$2$};
 \node at (-1.7,0.8) {${\it transferring}$};
  \node[shape=circle,draw=black,outer sep =0, inner sep=1] at (-2.4,2.5) {\small$3$};
 \node at (-1.6,2.5) {${\it taking}$};
\end{tikzpicture}
	\caption{Three phases of point-to-point communication.
	%\MS{I think this picture is very nice.}
	\label{Fig::schema}}
\end{figure}

Regarding these three phases, we consider a set of interface functions with well-defined responsibilities for actors, network and buffer entities through which they can interact to realize these phases. The behavior of each interface function may follow an specific policy to accomplish its responsibility. % for realizing each phase. % can be abstracted away by a set of interface functions on buffers specifying how messages are selected either for transferring or processing.
%: two belong to the abstract network entity, namely $\send$ and $\transfer$, two belong to (the buffer of) the actors, namely $\receive$ and $\take$. 
%Viewing actors, their buffers and the abstract network as an entity, 
The relation among entities and their functions are shown in Figure \ref{Fig::functions}. The sending phase starts when actors call their $\send$ function. This function has the responsibility to push the message into the abstract network by calling (asynchronously) its $\receive$ interface function. This function has the responsibility to store the received messages into the network buffer by calling the function $\it put$. The transferring phase starts when the abstract network entity calls its interface function $\transfer$ to dispatch received messages to its recipients. This function has the responsibility to first select a message among the received ones and then to deliver it to its recipient. This function works based on a policy that expresses when messages should be dispatched. It calls the function $\select$ of the buffer. When a message is selected for delivery, it is delivered to its destination actor by calling the $\receive$ function of actor. This function calls the $\put$ function of buffer that its behavior depends on the policy for maintaining the received messages. 
%Buffers are active entities. To receive messages, each buffer by calling the function $\it transfer$ of the network, gets the messages that are ready to be delivered to the buffer. They non-deterministically select one message among those messages to receive. 
The taking phase starts when actors call the function $\it take$ to process the messages of their buffers. This function handles messages selected based on the policy considered in the interface function $\select$ of the actor buffer. % gets the messages that are ready to be processed and may non-deterministically choose one message among those messages to handle.

\begin{figure}[htbp]
	\centering
	\begin{tikzpicture}[scale=0.8, transform shape]
%synchronous is full triangle
% -------------Actor left ---------------
\draw[line width=1pt]  (0,0) rectangle (2,1);
\node at (1,.5) {\small :$\it Actor$};
\draw (1,0) -- (1,-5);

\draw[line width=1pt]  (2.5,0) rectangle (4.5,1);
\node at (3.5,0.5) {\small :$\it Buffer$};
\draw (3.5,0) -- (3.5,-5);

%---------------Network --------------------
\draw[line width=1pt]  (5.5,0) rectangle (8.5,1);
\node at (7,0.5) {\small :$\it Abstract~ Network$};
\draw (7,0) -- (7,-5);
\draw[line width=1pt]  (9,0) rectangle (11,1);
\node at (10,0.5) {\small :$\it Buffer$};
\draw (10,0) -- (10,-5);
%---------------Actor right-------------------------------
\draw[line width=1pt]  (14.5,0) rectangle (16.5,1);
\node at (15.5,.5) {\small :$\it Actor$};
\draw (15.5,0) -- (15.5,-5);

\draw[line width=1pt]  (12,0) rectangle (14,1);
\node at (13,0.5) {\small :$\it Buffer$};
\draw (13,0) -- (13,-5);

%---------------sending--------------------
%Actor left layer
\node (s1) at (0,-0.7)  {};
\node (s2) at (1,-0.7) {};
\draw[->] (s1) edge[sloped,above] node{$\it send$} (s2);

\node (v1) at (1,-0.7)  {};
\node (v2) at (7,-1) {};
\draw[fill=white]  (0.9,-0.6) rectangle (1.1,-1);

%----------------------------------
%Network layer



\draw[->] (v1) edge[sloped,above] node{$\it receive$} (v2);
\draw[fill=white]  (6.9,-0.9) rectangle (7.1,-2.1);

\draw[fill=white]  (6.9,-2.3) rectangle (7.1,-3.9);
\node (r1) at (5.4,-2.5)  {};
\node (r2) at (6.9,-2.5) {};
\draw[->] (r1) edge[sloped,above] node{\small $\it transfer$} (r2);


%--------------Network buffer --------------------------------
\node (t1) at (7.1,-1.2)  {};
\node (t2) at (9.9,-1.4) {};
\draw[fill=white]  (9.9,-1.2) rectangle (10.1,-2.1);
\draw[-triangle 60] (t1) edge[sloped,above] node{$\it put$} (t2);

%\draw[-triangle 60] (7,-1.7) edge[loop right] node{$\it transfer$} (7,-1.7);
\draw[-triangle 60]  (10.1,-1.5) -- (10.4,-1.5) -- (10.4,-2) -- (10.1,-2);
\node (v3) at (11,-1.8)  {$\it insert$};
% node {$\it transfer$}(7.1,-1.8);
%\draw[fill=white]  (3.9,-2.1) rectangle (4.1,-2.5);

\node (t1) at (7.1,-2.5)  {};
\node (t2) at (9.9,-2.7) {};
\draw[fill=white]  (9.9,-2.5) rectangle (10.1,-3.4);
\draw[-triangle 60] (t1) edge[sloped,above] node{$\it select$} (t2);

%\draw[-triangle 60] (7,-1.7) edge[loop right] node{$\it transfer$} (7,-1.7);
\draw[-triangle 60]  (10.1,-2.7) -- (10.4,-2.7) -- (10.4,-3.2) -- (10.1,-3.2);
\node (v3) at (11,-2.95)  {$\it remove$};
% node {$\it transfer$}(7.1,-1.8);
%\draw[fill=white]  (3.9,-2.1) rectangle (4.1,-2.5);

\node (m1) at (9.9,-3.2)  {};
\node (m2) at (7.1,-3.4) {};
\draw[fill=white]  (9.9,-2.5) rectangle (10.1,-3.4);
\draw[dashed,->] (m1) edge[sloped,above] node{$\it msg$} (m2);
%-----------------------------------------------------------------

%Actor right
\draw[fill=white]  (15.4,-3.6) rectangle (15.6,-4.2);
\node (a1) at (7.1,-3.5)  {};
\node (a2) at (15.4,-3.8) {};
\draw[->] (a1) edge[sloped,above] node{$\it receive$} (a2);

\node (a1) at (15.4,-3.9)  {};
\node (a2) at (13.1,-4.1) {};
\draw[-triangle 60] (a1) edge[sloped,below] node{$\it put$} (a2);
\draw[-triangle 60]  (12.9,-4.2) -- (12.6,-4.2) -- (12.6,-4.7) -- (12.9,-4.7);
\node (v3) at (12,-4.45)  {$\it insert$};

%Actor buffer right
\draw[fill=white]  (12.9,-4) rectangle (13.1,-4.9);
%\draw[-triangle 60] (v3) edge[sloped,below] node{$\it receive$} (v4);

%\draw[->,dashed] (7,-2.4) edge[sloped, below] node{\small {msgs ready to be delivered}} (4,-2.7);

%-----------------buffer actor right
\draw[fill=white]  (3.4,-3.2) rectangle (3.6,-4.1);

\node (v5) at (1.1,-3)  {};
\node (v6) at (3.4,-3.3) {};
\draw[-triangle 60] (v5) edge[sloped,above] node{$\it select$} (v6);

\node (v7) at (3.4,-3.9)  {};
\node (v8) at (1.1,-4.2) {};
\draw[->,dashed] (v7) edge[sloped,above] node{\small $msg$} (v8);
\draw[-triangle 60]  (3.6,-3.3) -- (3.9,-3.3) -- (3.9,-3.8) -- (3.6,-3.8);
\node (v3) at (4.6,-3.6)  {$\it remove$};
%-------------------------------
%actor right
\draw[fill=white]  (0.9,-2.9) rectangle (1.1,-4.4);
\node (tt1) at (0,-3)  {};
\node (tt2) at (1,-3) {};
\draw[->] (tt1) edge[sloped,above] node{$\it take$} (tt2);

%\draw[<->,dashed] (1,-0.7) -- (1,-0.7) ;

\end{tikzpicture}
	\caption{The collaboration among the entities. The interface function $\send$ initiates the sending phase, $\transfer$ initiates the transferring phase, and $\take$ initiates the taking phase. %\FG{either the call from the send to transfer should be asynchronous or from the transfer to receive}
	%\MS{I am not sure, maybe Transfer is a loop back in the Network, causing the receive? 
	%I would draw it like this but it is not following any standards I guess: send causes transfer, transfer causes receive, receive causes take. Send goes from Actor to Network (like in the figure), Transfer is a loop back in the Network starting from the end of Send arrow, Receive starts from the end of Transfer arrow and comes back to the Buffer, Takes starts from somewhere from the buffer block after the end of the Receive arrow and goes to the Actor.
	%
	%This is my explanation:
	%An actor \textit{sends} a message, it is put in the Network Soup, the network \textit{transfers} it to the receiver. Here, receiving has some process: the Network  informs the receiver that the message is ``ready to be delivered", the receiver actor decides how to \textit{receive} it and put it in its buffer (like overwrite or overflow or keep the newest time-tag ...).
	%At some point the actor \textit{takes} it from the buffer based on a policy like  pattern-matching or an scheduling policy.}
	\label{Fig::functions}}
\end{figure}



\section{Semantic Specification Framework}\label{sec::generic}
In actor models, we distinguish three entities and we make them explicit: actors, buffers, and the network which we call it the abstract network. Our semantic specification framework provides a set of template rules to define the semantic model of actor-based languages. We identify a set of variation points in the semantic rules that their adjustment leads to different semantic rule implementations. These variation point explicitly shows the various assumptions and policies that are considered in designing/implementing the actor-based computation model. % based on Agha's actor model \cite{Agha90}. 
To make our framework independent of any specific language syntax, we provide a system abstract syntax model $\mathcal{M}$ that describes a specified system at a higher level of abstraction. Our abstract syntax model is defined as generic as possible to comply with most actor-based languages.

%Our generic framework defines the semantic model of a system abstract system model based on labeled transition systems (LTSs). 
Given a system abstract syntax model $\mathcal{M}$, we present its formal semantics as an LTS $\langle S,\rightarrow,s_0 \rangle$ where $S\subseteq \gstate$ %\subseteq \envT\times (\ID\rightarrow\local)\times {\it NetLocal}$ 
is the set of global states, also called system states. Global states are defined based on the local states of the actors and abstract network entity. To define the states of the abstract network entity, actors, and system, we consider an environment which is a mapping from variables to their values, called a valuation function. Assume $\Var$ is the set of variables and $\Value$ is the set of possible values that a variable is assigned. We define the set of valuation functions as  $\envT=\Var \rightarrow \Value$. Let $\domain(e)$ denote the domain variables of the environment while $e[\mathfrak{x}\mapsto\mathfrak{y}]$ denote updating the environment $e$ by mapping $\mathfrak{x}$ to $\mathfrak{y}$:\[
\begin{array}{l}
\forall a\in\domain(e)\cdot(a\neq\mathfrak{x}\Rightarrow e[\mathfrak{x}\mapsto\mathfrak{y}](a)=e(a) \wedge a=\mathfrak{x}\Rightarrow e[\mathfrak{x}\mapsto\mathfrak{y}](a)=\mathfrak{y}).
\end{array}
\]We use $e_1\cup e_2$ to aggregate the valuation functions $e_1$ and $e_2$ where the values assigned by $e_2$ overwrites the values of $e_1$ for common variables, where $\domain(e_1\cup e_2)=\domain(e_1)\cup\domain(e_2)$. We also use $e_1\setminus e_2$ to excludes valuations of $e_2$ from $e_1$, where $\domain(e_1\setminus e_2)=\domain(e_1)\setminus\domain(e_2)$:\[
\begin{array}{l} 
\forall a\in\domain(e_1 \cup e_2)\cdot(a\in \domain(e_2)\Rightarrow e_1\cup e_2(a) = e_2(a) \,\wedge \\ \hspace*{1cm} a\in \domain(e_1)\setminus\domain(e_2)\Rightarrow e_1\cup e_2(a) = e_1(a))\vspace*{1mm}\\
\forall a\in\domain(e_1 \setminus e_2)\cdot( e_1\setminus e_2(a) = e_1(a)) 
\end{array}
\]%We use $\Upsilon$ to show a valuation function that $\domain(\Upsilon)=\emptyset$.

The transition relation $\rightarrow \,\subseteq \gstate\times \Act \times \gstate$ expresses how the global states of the system evolve. We define the semantic rules of the system based on the behavior of the abstract network  %$\noverto{}\subseteq {\it NetLocal}\times\Act \times {\it NetLocal}$ 
and actors. %$\aoverto{}\subseteq \local\times \Act \times \local$. 
We express the behavior of actors and abstract network by providing template rules for their interface functions, i.e., $\receive$ and $\take$ of actors, and $\receive$ and $\transfer$ of abstract network entity. %a set of rules which show how their local states change. 
Both actor and abstract network rely on buffers, so we should also define the behavior of buffers upon calling their interface functions, i.e., $\put$ and $\select$. %We provide a template rule for each interface function of actors, i.e., $\receive$ and $\take$ and abstract network entity, i.e., $\receive$ and $\transfer$. %In addition to the rules for the interface functions, we define rules for the abstract network, actors, and system to specify how they internally progress. The meta-rules of the system subsume rules for the interactions among of its constituent entities.


%As the behavior of the abstract network and actors rely on buffer, we also propose the meta-rules for defining the behavior of buffer % $\boverto{} \subseteq \buffer\times \Act \times \buffer$ 
%upon calling their interface functions, i.e., $\put$ and $\select$. 
%The operational rules that derive the operational behavior of the system use the semantic rules of abstract network entity and actors. 
%The meta-rules of actors use the rules of buffer and statements to define the behavior of the actors standalone. We define the meta-rules for the interface functions of actors, i.e., $\receive$ and $\take$. The meta-rules of the abstract network define the behavior of network upon calling its interface functions, i.e., $\receive$ and $\transfer$. In addition to the rules for the interface functions, we define rules for the abstract network, actors, and system to specify how they internally progress. The meta-rules of the system subsume rules for the interactions among of its constituent entities.


\subsection{Abstract Syntax Model}\label{sec::AbstractSyntaxModel}
%To make our semantic model independent of any specific language syntax, we provide an abstract syntax model that describes specified models at a higher level of abstraction. We define this model with regard to Agha's actor model \cite{Agha90}.

Let $\ID$ be the set of actor identifiers, ranged over by $x$ and $y$, and $\Msg$ a set of messages communicated among the actors, ranged over by $m$. As the structure of message elements differs in each language, we do not specify it. The structure should be defined for each language as needed. %when the template rules are adjusted. 
We assume that each message has a name and can be accessed by the dot notation $m.\name$. We represent the set of message names by $\MName$. Actors handle messages based on the name of messages. Let $\Stmt$ denote the set of statements defined by the language.%. The elements of this set are defined by the language. %Each message is a pair of content and appended information, i.e,  $\Msg=\Content\times \InfoT$; the content is a triple of a sender identifier, method name, receiver identifier, i.e., $\ID\times \MName\times\ID$. The set $\MName$ is the set of method names that an actor can handle, ranged over by $\mathfrak{m}$. This set is defined by the given actor model. The type $\InfoT$ abstracts the additional information that are appended to the messages in different system settings; in the timed setting, it may contain the arrival time of messages and the message deadline or the priority of the message.  %\FG{In the basic framework, this type is empty.} 

\begin{defn}[Actor Abstract Syntax Model]\label{Def::absActor}
An abstract syntax model of an actor is defined by the vector $\langle id, V,\msgHandlr\rangle $ where $id\in \ID$ is the actor identifier, $v\subseteq \Var$ is the set of local variables, and $\msgHandlr : \MName \rightarrow \Stmt^*$ define the set of statement an actor should execute based on the message name%, and $\aqu\subseteq \ID$ is the set of acquaintances that an actor can communicate with.
\end{defn}
In the classic actor model of Agha, each actor is also defined by its acquaintances, the set of actors that it can communicate with. As this set may be dynamic in some languages \cite{mellati}, we remove it from the abstract syntax model and address it by considering a variable in the actor environment that maintains the set of acquaintance identifiers. 
%\fixme{aqu should be removed : in the semantics it has a meaning ; it was in the first actor models but dynamic}  

An actor-based system is composed of a set of actors and an abstract network. We define the abstract syntax model of the system based on its constituent actors. We remark that as the network entity is not explicitly specified in the most of actor-based languages, we do not define an abstract syntax model for network.  

\begin{defn}[System Abstract Syntax Model]\label{Def::absSystem}
An abstract syntax model of an actor-based system $\mathcal{M}$ is defined by the pair $\langle \ID, R\rangle $ where $R$ is the set of actor abstract syntax models with identifiers from $\ID$. %and $N$ is the abstract model of the network entity.
\end{defn}

%\fixme{what is the abstract model of the network, maybe we should define it with respect to LF?}


%------------------
\subsection{Template Rules of Buffer}
Buffer provides two interface functions $\select$ and $\put$ to collaborate with other entities. The function $\select$ includes the policy by which an item is selected from the buffer while the function $\put$ includes the policy by which an item is inserted into the buffer. %specifies the behavior of the buffer when an item is inserted into the buffer. 

The semantics of these interface functions depends on the inner structure of the buffer. Assume the data type $\buffer$ defines the inner structure buffers, ranged over by $b$. The template rules at this level define the meaning of interface function by the relation $\boverto{~~} \,\subseteq \buffer\times \Act \times \buffer$. The given rules have assumptions that may vary for each language and should be defined accordingly. 

Our framework considers a rule for each interface function: 
\[\begin{array}{cc}
\inferrule*[left = (Put)]{?{\it putPolicy}(b,m)}{b\boverto{m?}?\insert(b,m)}~~~~ &
~~~~\inferrule*[left = (Select)]{?{\it selectPolicy}(b,m)}{b\boverto{m!}?\remove(b,m)}
\end{array}
\] The goal of the assumption $?{\it putPolicy}(b,m)$ is to indicate when the message $m$ can be inserted into the buffer. By specifying this assumption, we determine the policy for handling overflow, or stale or duplicate messages. The goal of the assumption $?{\it selectPolicy}(b,m)$ is to express the conditions for removing an item from the buffer. By specifying this assumption, we explicitly identify the policy for selecting the message $m$ from the buffer. This policy may consider a priority or the timestamp of messages. The update of buffer $b$ upon inserting and removing an item depends on the structure of buffer. So the functions $?\insert(b,m)$ and $?\remove(b,m)$ should be defined on the data type $\buffer$.

%--------------------
\subsection{Template Rules of Actor}
We define the semantic of actors by the relation $\aoverto{}\,\subseteq \local\times \Act \times \local$ where $\local$ is the set of local states of actors. The local states of an actor are defined by the value of its variables, buffer content, and program counter. We represent a local state by the triple $(e,b,\sigma)$ where $e\in\envT_{\it Actor}$ is a valuation function, $b\in\buffer_{\it Actor}$ is the local buffer of the actor, and $\sigma\in \Stmt^*$ is the sequence of statements. It shows the program counter of an actor by expressing the remaining statements (from the program counter) of a message handler that should be executed. An empty sequence, denoted by $\epsilon$, shows that the actor is not busy by executing a message handler. The set of actor local states is defined as $\local = \envT_{\it Actor}\times\buffer_{\it Actor}\times 
\Stmt^\ast$. 

The rules of actors are defined in two levels: statement and actor level. The latter specifies the semantics of the actor function interfaces, namely $\receive$ and $\take$, and how the actor internally evolves. The network calls the interface function $\receive$ to deliver a message to the actor.  The interface function $\take$ aims at processing a message from the buffer. This interface is invoked by the thread of each actor and includes the actor-level policy for choosing a message from the actor buffer. 

We consider four actor-level meta-rules; two rules for the interface functions and two for internal progress of the actor:  
\begin{itemize}
    \item The rule \textsc{Receive} specifies the semantics of the interface function $\receive$. This rule explains that an actor can always receive a message as long as its buffer accepts an item.
\[
\inferrule*[left = (Receive)]{b\boverto{m?}b'}{(e,b,\sigma)\aoverto{m?}(e,b',\sigma)}
\] 
    \item The rule \textsc{Take} defines the semantics of the interface function $\take$, and shows that a message is selected from the buffer when the actor is not busy with handling another message. The actor processes the selected message by executing the statements of its message handler, i.e., $\msgHandlr(m.\name)$. 
\[
\inferrule*[left = (Take)]{b\boverto{m!}b'%\\?{\it takePolicy}(e,b)
}{(e,b,\epsilon)\aoverto{m}(e,b',\msgHandlr(m.\name))}
\]%The rule has an assumption, denoted by $?{\it takePolicy}(e,b)$, for defining the extra actor-level policy needed for selecting a message. {\color{red}For instance, an assumption may enforce a pattern on the message to be processed.} We can implement the selection policy either at the buffer-level (by specifying $?{\it takePolicy}$ assumption for the rule \textsc{Select}) or the actor-level. The actor-level policy may adapt the selection based on the state of the actor. \fixme{Validate this and add example}
%in cases that each actor has a single thread of execution and its thread can be suspended, this assumption may express that the actor is ready to take a message if it is not suspended. 

\item Actors internally progress due to execution of their message handler's statements as explained by the rule \textsc{Internal}. By executing statements, the values of variables may update. As explained by the rule, executing the statements has no effect on the buffer. % evolves the   or updating the local time of the actor.
%\fixme{notSuspended should be removed}
\[
\inferrule*[left = (Internal)]{e,\sigma\stoverto{\tau}e',\sigma'}%\\?{\it notSuspended}(e)}
{(e,b,\sigma)\aoverto{\tau}(e',b,\sigma')}%:{\it stProgress}
%\inferrule*[left = (Internal)]{?{\it intCondition}(e,\sigma)}{(e,b,\sigma)\aoverto{\alpha}(?e',b,?\sigma')}%:{\it internal}
\]%where $\alpha=\{m!,\tau\}$. 
%The rule has a hole, denoted by $?{\it notSuspended}(e)$, for defining the extra conditions needed for executing its statements. 
%For example an actor can have internal progress if it has some statement to execute and it is not suspended. 
%This assumption is required in settings that the execution of actor can be temporarily suspended. We remark that internal progress of an actor has no effect on the buffer of the actor. %This rule also has two other holds $?e'$ and $?\sigma'$ which should be defined with regard to the type of the actor internal progress.

%\fixme{Make sure that this "notsuspended" is the case for all cases ...}

\item In actor-based languages with the concept of global time, the global time evolves in agreement with actors. The rule \textsc{EnvSybc} makes the progress of times explicit for such languages. This rule synchronizes with the rule at the system level advancing the global time.
\[
\inferrule*[left = (EnvSync)]{?{\it synConditon}(e,b,t)}{(e,b,\sigma)\aoverto{t}(e,b,\sigma)}
\]The rule has an assumption, denoted by $?{\it synConditon}(e,b,t)$, for identifying conditions to indicate when time with  the amount of $t\in \int$ can progress from the point view of the actor. \fixme{example needed} %It also indicate that it agrees with  the amount of $t\in \int$ to be advanced by the global environment.
%\fixme{Only when we have global time concept, then ...}
\end{itemize}


The statement-level rules should be provided by the user with regard to the set of statements. The meta-rule for each statement $s\in\Stmt$ is defined as:
\[
\inferrule*{?{\it stCondition(e,s)}}{(e,s)\stoverto{\tau}(?{\it UpdEnv}(e,s),?{\it Continue}(e,s))}
\]where the assumption $?{\it stCondition(e,s)}$ expresses conditions on the environment $e$ and the structure of $s$. It should also define how the environment is updated and how the execution continues by specifying the two effect functions $?{\it UpdEnv}(e,s)$ and $?{\it Continue}(e,s)$. 
%\[
%\Rule{?{\it stCondition(e,s)}}{(e,s)\overto{\tau}(e',\top)}
%\]where the assumption $?{\it stCondition(e)}$ may express condition on the environment $e$. This rule explains when a single statement is successfully executed, denoted by $\top$. For compound statements $s(\sigma_1,\ldots,\sigma_m)\in\Stmt$, where $\sigma_i\in\Stmt^*$, their meta-rules are defined as 
%\[
%\Rule{?{\it stCondition(e,\sigma_1,\ldots,\sigma_m)}}{(e,s(\sigma_1,\ldots,\sigma_m))\overto{\tau}(e',\sigma')}
%\] 
%-----------------------
\subsection{Template Rules of Abstract Network}
We define the semantics of the abstract network entity by the relation $\noverto{}\,\subseteq {\it NetLocal}\times\Act \times {\it NetLocal}$ where ${\it NetLocal}$ is the set of network entity local states. We represent the local state of the abstract network entity by a pair of $(e,b)$ where $e\in\envT_{\it Net}$ is the valuation of network local variables and $b\in\buffer_{\it Net}$ is its local buffer.  The set of local states of the network is defined as ${\it NetLocal}=\envT_{\it Net}\times \buffer_{\it Net}$.

The abstract network entity provides two interface functions $\receive$ and $\transfer$. An actor calls the interface function $\receive$ to deliver its message for transmission. The interface function $\transfer$ is called to transfer the buffer messages. (If we consider the abstract network entity as an active entity, the thread of network calls $\transfer$.) The network-level meta-rules define the semantics of the network function interfaces and how it internally progresses:
\begin{itemize}
    \item The rule \textsc{Receive} specifies the semantics of interface function $\receive$. This rule explains that the network can always receive a message as long as its buffer accepts an item.
\[
\inferrule*[left = (Receive)]{b\boverto{m?}b'}{(e,b)\noverto{m?}(e,b')}
\] 
    \item The rule \textsc{Transfer} defines the semantics of the interface function $\transfer$. 
    \[
\inferrule*[left = (Transfer)]{b\boverto{m!}b'\\ ?{\it transferPolicy(e,b,m)}}{(e,b)\noverto{m!}(e,b')}%:{\transfer}
\]The rule has an assumption $?{\it transferPolicy(e,b,m)}$.  By specifying this assumption, we explicitly identify the policy on transmitting messages. For instance, this policy may enforce the order of messages should be preserved by their delivery. In timed languages, the policy may transfer messages based on their timestamps. For messages with the same timestamp, the policy may consider a priority among messages or transfer them non-deterministically. \fixme{By this assumption, we can also indicate conditions on when the network is ready to transfer a message. validate?}     
%\fixme{From actor a to actor b the order of messages is preserved. when a message should }

\item In languages with the concept of global time, the global time evolves in agreement with the abstract network entity. The rule \textsc{EnvSybc} synchronizes with the rule at the system level advancing the global time.
\[
\inferrule*[left = (EnvSync)]{?{\it synCondition}(e,b,t)}{(e,b,\sigma)\noverto{t}(e,b,\sigma)}%:{\it envSync}
\]The rule has an assumption, denoted by $?{\it synConditon}(e,b,t)$, for identifying conditions to indicate when time with  the amount of $t\in \int$ can progress from the point view of  the abstract network entity.  %It also indicate that it agrees with  the amount of $t\in \int$ to be advanced by the global environment.
\end{itemize}
%-----------------------------------

\subsection{Template Rules of the System}
%\FG{System is composition rule!}
We define the semantics of the system by the relation $\overto{}\,\subseteq \gstate\times\Act \gstate $ where $\gstate$ is the set of global states. The global states of the system are defined by the values of global variables, the local states of actors, and the local state of the abstract network entity. We represent a global state by the triple $(e,s,n)$ where $e\in\envT_{\it Sys}$ is the valuation function for the global variables that are shared among the actors, $s: \ID\rightarrow \local$ is a mapping from actor identifiers to their local state, and $n\in {\it NetLocal}$ is the network local state. %The environment $e$ defines a set  Actors cannot change these variables and can only read from them.

The rules at this level define the compositional semantics of the actor-based systems. The rules define the semantic of collaboration among the system constituents, i.e., actors and abstract network or the evolution of the system in terms of its constituent evolution.  

\begin{itemize}
    \item The global state evolves when an actor internally progresses, as explained by the following rule.
    \[
\inferrule*[left = (actorProgress)]{s(x)==(e^\ast,b,\sigma) \\ (e\cup e^\ast, b, \sigma)\aoverto{\tau}(e',b',\sigma')}{(e,s,n)\overto{\tau}(e,s[x\mapsto(e'\setminus e,b',\sigma')],n)}%:{\it actorProgress}
\] 

\item The global state evolves when an actor delivers a message for transmission to the abstract network entity. The assumption $(e\cup e^\ast, b, \sigma)\aoverto{m!}(e',b',\sigma')$ indicates that the actor with the identifier $x$ has the message $m$ for delivery (by executing a statement generating a message). In this assumption, the actor is executed under its environment extended with the global environment. The assumption $(e\cup e^{\ast\ast},b^{\ast\ast})\noverto{m?}(e'',b'')$ expresses that the abstract network entity is ready to accept the message $m$. This assumption is the conclusion of \textsc{Receive} rule of the network-level rule. 
    \[
\inferrule*[left = (comm 1)]{s(x)==(e^\ast,b,\sigma)\\(e\cup e^\ast, b, \sigma)\aoverto{m!}(e',b',\sigma')\\n==(e^{\ast\ast},b^{\ast\ast})\\(e\cup e^{\ast\ast},b^{\ast\ast})\noverto{m?}(e'',b'')}{(e,s,n)\overto{\gamma}(e,s[x\mapsto(e'\setminus e,b',\sigma')],(e''\setminus e,b''))}%:{\it comm_1}
\] 

\item This rules shows the communication of the abstract network entity with an actor. When the network has a message ready to deliver, indicated by the assumption $(e\cup e^{\ast\ast},b^{\ast\ast})\noverto{m!}(e'',b'')$ and an actor is ready to receive, expressed by  the assumption $(e\cup e^\ast, b, \sigma)\aoverto{m?}(e',b',\sigma')$, the message is transferred. 
    \[
\inferrule*[left = (comm 2)]{?{\it PriorityPolicy}(e,s,n)\\n==(e^{\ast\ast},b^{\ast\ast})\\(e\cup e^{\ast\ast},b^{\ast\ast})\noverto{m!}(e'',b'')\\m.rcv==y \\ s(y)==(e^\ast,b,\sigma)\\(e\cup e^\ast, b, \sigma)\aoverto{m?}(e',b',\sigma')}{(e,s,n)\overto{\gamma}(e,s[x\mapsto(e'\setminus e,b',\sigma')],(e''\setminus e,b''))}%:{\it comm_2}
\]This rule has an assumption $?{\it PriorityPolicy}(e,s,n)$. By specifying this assumption, we can define the policy that considers a priority among the entities. For instance, we can define the policies of CAN bus for the abstract network entity. As CAN bus transfers messages with the same timestamp based on their the priority defined on the message types, the network should not start transferring messages as long as actors have messages to deliver the network. By giving a lower priority to the network in comparison with actors (and defining $?{\it transferPolicy$ of rule \textsc{Transfer} at the network-level), the network entity behaves as CAN bus.  

\item In languages with the concept of global time, this rule explicitly explains when the global time advances. This rule synchronizes with the rule \textsc{envSync} of entities of systems, i.e., actors and the abstract network. %The assumption $?{\it envUpd}(e,s,n,t)$ shows the conditions on .
\[
\inferrule*[left = (envProgress)]{?{\it envUpd}(e,s,n,t)}{(e,s,n)\overto{t}(e[\now \mapsto t],s,n)}%:{\it envProgress}
\]where $t\in\int$ is the new value of the global time. The assumption $?{\it envUpd}(e,s,n,t)$ indicates the conditions or advancing the global time to $t\in \int$.
\end{itemize}

We have summarized the meta-rules for each level with their parametric assumptions in Table \ref{Tab::semanticPar}. To define the semantic rules of a language, we only instantiate those rules by specifying their parametric assumptions. The rules with no parametric assumption are inherited from the generic framework with no modification. 



\begin{table}[]
	\centering
	\caption{The parameters of rules at each level that should be defined}\label{Tab::semanticPar}
	\begin{tabular}{|c|c|l|l|l|}
		\hline
		\multirow{2}{*}{Level}  & \multirow{2}{*}{Relation} & \multicolumn{3}{l|}{Rule}                     \\ \cline{3-5} 
		&                           & Name                    & Parameter        & Signature \\ \hline
		\multirow{4}{*}{Buffer} & \multirow{4}{*}{$\boverto{\;\;\;}$}         & \multirow{2}{*}{\textsc{put}}    & $?{\it putPolicy}$    &  $\buffer\times \Msg\rightarrow \bool$    \\ \cline{4-5} 
		&                           &                         & $?\insert$       &   $\buffer\times \Msg\rightarrow\buffer$   \\ \cline{3-5} 
		&                           & \multirow{2}{*}{\textsc{select}} & $?{\it selectPolicy}$ &    $\buffer\times \Msg\rightarrow \bool$  \\ \cline{4-5} 
		&                           &                         & $?\remove$       &  $\buffer\times \Msg\rightarrow\buffer$    \\ \hline
		%------------------------------------------------------------------
		\multirow{7}{*}{Actor}  & \multirow{7}{*}{$\aoverto{}$}         & \textsc{Receive}                 &     -         &  -    \\ \cline{3-5} 
		&                           & \textsc{Take}                    &    -          & $\envT\times \buffer\rightarrow \bool$     \\ \cline{3-5} 
		&                           & \textsc{Internal}                &    -         &  -   \\ \cline{3-5} 
		&                           & \textsc{EnvSync}                 &    $?{\it synCondition}$         & $\envT\times \buffer\times \int\rightarrow \bool$    \\ \cline{3-5} 
		&                           & \multirow{3}{*}{$s\in\Stmt$}   & $?{\it stCondition}$ &  $\envT \times \Stmt\rightarrow \bool$    \\ \cline{4-5} 
		&                           &                         & $?{\it updEnv}$       &  $\envT \times \Stmt\rightarrow \envT$    \\ \cline{4-5} 
		&                           &                         & $?{\it Continue}$     &  $\envT \times \Stmt\rightarrow \Stmt$    \\ \hline
		%------------------------------------------------------------------
		\multirow{3}{*}{Network} & \multirow{3}{*}{$\noverto{}$}         & \textsc{Receive}                 &       -       &    -  \\ \cline{3-5} 
		&                           & \textsc{Transfer}                &      $?{\it transferPolicy}$        &  $\envT\times \buffer\times \Msg\rightarrow \bool$    \\ \cline{3-5} 
		&                           & \textsc{EnvSync}                 &       $?{\it synConditon}$         &   $\envT\times \buffer\times \int\rightarrow \bool$   \\ \hline
		%---------------------------------------------------------------------------
		\multirow{4}{*}{System} & \multirow{4}{*}{$\overto{}$}         & \textsc{ActorProgress}           &      -        &    -  \\ \cline{3-5} 
		&                           & \textsc{Comm 1}                  &      -        &   -   \\ \cline{3-5} 
		&                           & \textsc{Comm 2}                  &     $?{\it PriorityPolicy}$         &   $(\ID\rightarrow \local)\rightarrow \bool$   \\ \cline{3-5} 
		&                           & \textsc{EnvProgress}             &     $?{\it envUpd}$        &   $\gstate\times \int\rightarrow \bool $   \\ \hline
	\end{tabular}
	
\end{table}
\section{Actor model Variations}\label{sec::combine}
To illustrate the applicability of our framework, we define the formal semantics of actor-based languages by defining the types for defining the structure of buffers and message, and by specifying the parametric assumptions of the rules summarized in Table \ref{Tab::semanticPar}. The rules with no parametric assumption are inherited from the generic framework with no modification. For each language, we only consider those statements that do not limit the asynchronous communication like wait or callback statements. We define the structure of messages, $\buffer$s and $\envT$s for each level using abstract data types. %For the types $\buffer_{\it Actor}$ and $\buffer_{\it Net}$, the assumptions $?{\it putPolicy}(b)$ and $?{\it selectPolicy}(b)$ should be defined.

We first provide the semantic rules of the Rebeca. Due to its design principle it is possible to extend the core language based on the desired domain \cite{sirjani2011ten}. We also go through its extension for the real-time domain, Timed Rebeca and then Hybrid Rebeca for cyber physical systems. Hybrid Rebeca extends Timed Rebeca with continuous behaviors, we show that Hybrid Rebeca inherits the rules of Timed Rebeca and so its semantic rules can be easily achieved. After Rebeca family, we focus on ABS and Lingua Franca.

%The semantic rules of each actor-based language can be explicitly defined in terms of the meta-rules of the generic framework by specifying the structure of $\buffer$s and $\envT$s and unknown assumptions $?$ as given in  


%In following subsections  to illustrate the applicability of our framework. 

\subsection{Core Rebeca}
Rebeca \cite{sirjani2004modeling} has a Java-like syntax, familiar to software developers, and it is also supported by a tool via an integrated modeling and verification environment \cite{afra}.  We briefly introduce the syntax of Rebeca. A Rebeca model consists of a set of class declarations and a main block. Actors, called rebecs, are instances of the defined \emph{reactive classes} in the model. Each class has three parts: \emph{state variables}, \emph{known rebecs}, and \emph{message servers}. Each class with the name $C$ has one message server called $C$, which acts like a constructor in object-oriented languages and performs the initialization tasks. The main block expresses the instanced rebecs; the  known rebecs of each actor and initial values of variables are passed via instantiating. 
\begin{figure}[h]
	\centering
	\lstinputlisting[language=HRebeca, multicols=2]{"./Code/rebec.tex"}
	\caption{A model of a monitor and an alarm specified in  Rebeca. The commented statements belong to Timed Rebeca.}
	\label{fig:Controller}
\end{figure}

\begin{example}
The Rebeca model given in Figure \ref{fig:Controller} shows a system composed of a monitor and an alarm. The monitor may be informed from the temperature of its environment by receiving a $\it check$ message. If the message is above the specified value $\it max$, it sends a $\it notify$ message to its known rebec of class $\it Alarm$ (line 19). In the main block, the known rebecs are passed via the first pair of parenthesis and actual value of constructors via the second pair of parenthesis. As the instance $\it alarm$ is the known rebec of instance $\it monitor$, it is first passed in via the first pair of parenthesis in line 37 and then initial value for $\it max$, i.e., $25$ via the second pair of parenthesis. 
\end{example}

Each Rebeca model is trivially converted into a system abstract syntax model. For simplicity, we have assumed that messages has no parameter. There is a mapping between the concepts of Rebeca model and the system abstract syntax model. The set of $\ID$ is defined by the name of instantiated rebecs. For each rebec of class $C$, the class message servers are $\msgHandlr$ of the rebec, class state variables are $\Var$. %while class known rebecs are its $\aqu$. 
The names of all message servers determine the set of $\MName$. The set of $\Stmt$ for defining the body of message serves includes:\begin{itemize}
    \item send statement $x!\mathfrak{m}$ that sends a message with the name $\mathfrak{m}\in\MName$ to the rebec with the identifier $x$.
    \item assignment $v={\it expr}$ that assigns the value of the expression ${\it expr}$ to $v\in\Var$;
    \item sequential composition $s_1;s_2$ that makes the two statements $s_1$ and $s_2$ execute sequentially.
    \item conditional statement ${\it if}~ (\expr) ~s_1 ~{\it else}~s_2 $ that executes the statement $s_1$ and $s_2$ based on the Boolean value of $\expr$,
\end{itemize}

\subsubsection{Structure of Messages and Buffers}
%We define the structure of actor buffers as an unbounded Fifo queue of messages $\buffer_{\it Actor}=\Fifo (\Msg)$. Assume the following functions on $\Fifo(D)$:\begin{itemize} 
%\item $\frown:\Fifo(D)\times D\rightarrow \Fifo(D)$: specifies a queue that the given item has been added to the end of the given Fifo queue;
%\item $\tail:\Fifo(D)\rightarrow \Fifo(D)$: specifies a queue that the head of given Fifo has removed;
%\item $\head:\Fifo(D)\rightarrow D$: specifies the head element of the given queue.
%end{itemize}
The messages in this setting consists of three parts: sender identifier, message name, and receiver identifier. So the set of messages is defined as $\Msg = \ID\times\MName\times \ID$. We use dot notation $m.\rcv$ to access the receiver identifier.  

We define the structure of actor buffers as an unbounded FIFO queue of messages. % by the two constructors $\empt: \buffer_{\it Actor}$, denoting an empty buffer, and $\frown: \Msg \times \buffer_{\it Actor}  \rightarrow \buffer_{\it Actor}$, appends a message to the buffer. 
We define the following functions for manipulation of $\buffer_{\it Actor}$:\begin{itemize} 
\item $\insert:\buffer_{\it Actor} \times \Msg\rightarrow \buffer_{\it Actor}$: given a buffer and message, this function inserts the message into the buffer;
\item $\head:\buffer_{\it Actor} \rightarrow \Msg$: specifies the head element of the given queue;
\item $\remove:\buffer_{\it Actor} \times \Msg \rightarrow \buffer_{\it Actor} $: if the given message is the head of the queue, it returns a queue by removing the head of the queue;
\item $\size:\buffer_{\it Actor} \rightarrow \int$: denotes the number of elements in the buffer.
\end{itemize}



As the order of delivery of messages matches with the order of their sending for each actor, we define the structure of $\buffer_{\it Net}$ by a list of actor identifiers and their pending message queue. % by the two constructors $\empt: \buffer_{\it Net}$, denoting an empty set, and $\triangleright: \ID \times \buffer_{\it Actor} \times \buffer_{\it Net} \rightarrow \buffer_{\it Net}$. 
We define the following functions on $\buffer_{\it Net}$:\begin{itemize} 
\item $\insert:\buffer_{\it Net} \times \Msg\rightarrow \buffer_{\it Net}$: given a buffer and message, this function inserts the message into the queue of the receiving actor specified in the message.;
%\item $\head:\buffer_{\it Actor} \rightarrow \Msg$: specifies the head element of the given queue;
\item $\remove:\buffer_{\it Net} \times \Msg \rightarrow \buffer_{\it Net} $: it removes the given message from the queue of the receiving actor specified in the message,
\item ${\it getBuff}:\buffer_{\it Net} \times \ID \rightarrow \buffer_{\it Actor} $: Given a buffer and an actor identifier, this function returns the actor buffer of actor.
%\item $\size:\buffer_{\it Actor} \rightarrow \int$: denotes the number of elements in the buffer.
\end{itemize}

The formal description of buffer structures and their functions are given in Section \ref{sec::dataCore}.

\subsubsection{Semantic Rules}
The semantic rules derived from the generic framework are given in Tables \ref{Tab::ClassicRules}. %The assumptions of the rules are summerized in Table \ref{Tab::ClassicAssm}. 
%\in \envT_{\it Actor}




\begin{table}[]
\centering
\caption{The semantic rules of the classic Rebeca: $q\in \buffer_{\it Actor}$ and $b\in \buffer_{\it Net}$.}
\label{Tab::ClassicRules}
\begin{tabular}{|l|ccc|}
\hline
\multirow{2}{*}{\begin{sideways}Buffer\end{sideways}} & \multicolumn{1}{c|} {\begin{sideways}Actor\end{sideways}}&   $\inferrule*[left = (Select)]{\size(q)>0\\m==\head(q)}{q\boverto{m!}\remove(q,m)}$ & $\inferrule*[left = (Put)]{}{q\boverto{m?}\insert(q,m)}$\\[1mm] \cline{2-4} 
& {\begin{sideways}Network\end{sideways}} & \multicolumn{2}{|c|}{$\inferrule*[left = (put)]{}{b\boverto{m?}\insert(b,m)}$}\\ [1mm]
& & \multicolumn{2}{|c|}{$\inferrule*[left = (select)]{\exists x\in\ID \cdot \size({\it getBuff}(b,x))>0\\m==\head({\it getBuff}(b,x))}{b\boverto{m!}\remove(b,m)}$} \\[1mm]
\hline
%------------Actor
\multirow{5}{*}{\begin{sideways}Actor\end{sideways}} & & $\inferrule*[left = ( Receive)]{q\boverto{m?}q'}{(e,q,\sigma)\aoverto{m?}(e,q',\sigma)}$ & 
$\inferrule*[left = (Internal)]{e,\sigma\stoverto{\tau}e',\sigma'}
{(e,q,\sigma)\aoverto{\tau}(e',q,\sigma')}$ \\[1mm]
& \multicolumn{3}{c|}{$\inferrule*[left = (Take)]{q\boverto{m!}q'}{(e,q,\epsilon)\aoverto{m}(e,q',\msgHandlr(m.\name))}$} \\[1mm] \cline{2-4}
%---------------Statement
& \multirow{3}{*}{\begin{sideways}Statement\end{sideways}} &
\multicolumn{1}{|c}{$\inferrule*[left = (Cond 1)]{\eval(\expr,e)\\e,\sigma_1\stoverto{\tau}e',\top}{e,{\it if}~ (\expr) ~\sigma_1 ~{\it else}~\sigma_2 \stoverto{\tau}e',\top}$} & $\inferrule*[left = (Seq)]{e,\sigma_1\stoverto{\tau}e,\top}{e,\sigma_1;\sigma_2\stoverto{\tau}e,\sigma_2}$\\[1mm]
& &  \multicolumn{1}{|c}{$\inferrule*[left = (Cond 2)]{\neg\eval(\expr,e)\\e,\sigma_2\stoverto{\tau}e',\top}{e,{\it if}~ (\expr)~ \sigma_1 ~{\it else}~\sigma_2 \stoverto{\tau}e',\top}$} & $\inferrule*[left = (Send)]{}{e,y!\mathfrak{m} \stoverto{(e({\it self}),\mathfrak{m},y)!} e,\top}$\\[1mm]
& & \multicolumn{2}{|c|}{$\inferrule*[left = (Assign)]{}{e,v= \expr \stoverto{\tau} e[v\mapsto \eval(\expr,e)],\top}$} \\[1mm]
\hline
%-------------Network
{\begin{sideways}Network\end{sideways} & & $\inferrule*[left = (Receive)]{b\boverto{m?}b'}{(e,b)\noverto{m?}(e,b')}$   & $\inferrule*[left = (Transfer)]{b\boverto{m!}b'}{(e,b)\noverto{m!}(e,b')}$ \\[1mm] 
\hline
%---------------System
\multirow{3}{*}{\begin{sideways}System\end{sideways}} &  \multicolumn{3}{c|}{$\inferrule*[left = (actorProgress)]{s(x)==(e^\ast,b,\sigma) \\ (e^\ast, b, \sigma)\aoverto{\tau}(e',b',\sigma')}{(e,s,n)\overto{\tau}(e,s[x\mapsto(e',b',\sigma')],n)}$}\\[1mm]
&\multicolumn{3}{c|}{$\inferrule*[left = (comm 1)]{s(x)==(e^\ast,q,\sigma)\\(e^\ast, q, \sigma)\aoverto{m!}(e',q',\sigma')\\ n \noverto{m?}n'}{(e,s,n)\overto{\gamma}(e,s[x\mapsto(e',q',\sigma')],n')}$}\\[1mm]
&\multicolumn{3}{c|}{$\inferrule*[left = (comm 2)]{n\noverto{m!}n'\\m.rcv==x \\ s(x)==(e^\ast,q,\sigma)\\(e^\ast, q, \sigma)\aoverto{m?}(e',q',\sigma')}{(e,s,n)\overto{\gamma}(e,s[x\mapsto(e',q',\sigma')],n')}$}\\
\hline
\end{tabular}
\end{table}

As the queue of actor buffers are unbounded, the interface function $\put$ has no limitation and its $?{\it putPolicy}$ is set to true. A Message can be selected from a queue, if there is at least one message in the queue. Furthermore, the actor buffer selection policy chooses the message at the head of the queue. Similar to the actor buffers, the pending message queue for each rebec is unbounded, so the interface function $\put$ for the network buffer has no limitation and its $?{\it putPolicy}$ is also set to true. The selection policy from the pending message queue of each rebec is similar to the actor buffers. It should be noted that network buffer non-deterministically chooses an actor to take from its pending message queue.


The local state of an actor, specified by $\langle id, V,\msgHandlr\rangle$, is defined by the triple $(e,b,\sigma)$ where $e$ defines the value of the state variables of the actor such that $V\cup\{\self\}=\domain(e)$ and $e(\self)=id$. Rebecs can send message to themselves by using the reserved word $\it self$. The classic Rebeca has three actor-level rules \textsc{Receive}, \textsc{Take}, and \textsc{Internal} for showing the behavior of actor, and five rules for its statements. The rules \textsc{Receive} and \textsc{Internal} of the generic framework has no assumption, and we only define the assumptions of the rule \textsc{Take} and those of the statements. In this setting, the take policy handles messages in the order of their arrival. As the select function interface of an actor buffer chooses a message from the head of the buffer, this policy is enforced by the select policy. So the assumption $?{\it takePolicy}(e,b)$ is set to $\true$. The rule \textsc{Send} defines the behavior for the send statement, the rule \textsc{Assign} for assignment, and the rule \textsc{Seq} for the sequential composition. In the rule \textsc{Assign}, $\eval(\expr,e)$ is the value of the expression $\expr$ with respect to the environment $e$. The two rules \textsc{Cond 1} and \textsc{Cond 2} express the behavior of the conditional statement. 

The domain of environment for the network is empty. The classic Rebeca has two network-level rules \textsc{Receive} and \textsc{Transfer}. %As this setting has no time concept, there is no need to define the rule \textsc{EnvSync}. So we should only define the assumption $?{\it transferPolicy(e,b)}$. 
The abstract network policy for dispatching the pending messages of each rebec based on their arrival. As the pending messages of each rebec are maintained based on their arrival in the data structure of the network buffer, $?{\it transferPolicy(e,b)}$ is set to $\true$ in the rule \textsc{Transfer}.

The domain of environment for the system is empty. As for any $e^\ast\in\envT$, $e^\ast\cup \e=e$ and $e^\ast\setminus e=e$ when $\domain(e)=\emptyset$, the rules at the system level are simplified as illustrated in Table \ref{Tab::ClassicRules}. Since no priority is considered for the delivery of messages, the assumption $?{\it priorityPolicy}$ is set to true in the rule \textsc{Comm 2}. \fixme{Explain this more!} %As this setting has no time concept, there is no need to define the rule \textsc{EnvSync}.





%We have summarized how the assumptions of rules have been defined. 





\subsection{Timed Rebeca}
Timed Rebeca extends core Rebeca by adding time-passing statements:\begin{itemize}
    \item send statement $x!\mathfrak{m} ~{\it after} ~\expr$ that sends a message with the name $\mathfrak{m}\in\MName$ to the actor with the identifier $x$ after an specified delay that its value is defined by $\expr$. This statement is used to model the delay of network.
    \item delay ${\it delay}({\it expr})$ imposes an specified delay that its value is defined by $\expr$. This statement is used to model computation times. 
\end{itemize}

\begin{example}
Uncommenting the statements in Figure \ref{fig:Controller} results a model specified in Timed Rebeca. In this model, upon receiving a $\it check$ message, the monitor examines the temperature after a delay of $2$ (line 17). However, if a  $\it notify$ message is sent by the monitor to the alarm, the message is delivered to the alarm after the network delay of $1$.
\end{example}


\subsubsection{Structure of Messages and Buffers}
The messages in this setting have four parts: sender identifier, message name, receiver identifier, and arrival time. So the set of messages is defined as $\Msg = \ID\times\MName\times \ID\times \Nat$. We use dot notation $m.\rcv$ and $m.\ar$ to access the receiver identifier and the message arrival part. 


In this setting the structure of actor and network buffers are identical and it is defined as an unbounded bag of messages. %We define bag by the two constructors $\empt: \buffer$, denoting an empty bag, and $\diamond: \Msg \times \buffer\rightarrow \buffer$, adds a message to the buffer. 
We define the following functions on $\buffer$:\begin{itemize} 
\item $\insert:\buffer \times \Msg\rightarrow \buffer$: given a buffer and message, this function inserts the message into the buffer;
%\item $\head:\buffer \rightarrow \Msg$: specifies the head element of the given queue;
\item $\remove:\buffer \times \Msg \rightarrow \buffer $: if the given message is included in the bag, this function removes the message from the bag. Otherwise, the bag is not changed;
\item $\include:\Msg\times \buffer \rightarrow \bool$: its value is true if the given bag contains the given message. Otherwise, its value is false.
\item ${\it isEmpty}:\buffer\rightarrow \bool$: returns true if the given bag contains at least a message.
\end{itemize}

%In this setting the structure of actor buffers and network buffer are identical and are defined as an unbounded bag of messages $\buffer=\bag(\Msg)$. Assume the following functions on $\bag(D)$:
%\begin{itemize} 
%\item $\oplus:\bag(D)\times D\rightarrow \bag(D)$: specifies a bag that the given item has been inserted into the given bag;
%\item $\ominus:\bag(D)\times D\rightarrow \bag(D)$: specifies a bag that the given item has been removed from the given bag;
%\item $\include:D\times \bag(\Msg)\rightarrow \bool$: its value is true if the given bag includes the given item. Otherwise, its value is false.
%\end{itemize}

\subsubsection{Semantic Rules}
The semantic rules derived from the generic framework are given in Table \ref{Tab::TimedRules}. As this setting is timed, we should define the rule \textsc{EnvSync} for the actor and network level, and \textsc{EnvProgress} for the system level. %\fixme{when time can pass}



\begin{table}[]
\centering
\caption{The semantic rules of the Timed Rebeca.}
\label{Tab::TimedRules}
\begin{tabular}{|c|lcc|}
\hline
\multirow{2}{*}{\begin{sideways}Buffer\end{sideways}} & \multicolumn{3}{c|}{$\inferrule*[left = (put)]{}{b\boverto{m?}\insert(b,m) }$}\\ [1mm]
&   \multicolumn{3}{c|}{$\inferrule*[left = (Select)]{\include(m,b)\\\forall m'\cdot (\include(m',b)\Rightarrow m.\ar\le m'.\ar)}{b\boverto{m!}\remove(b,m) }$}\\[1mm] 
\hline
%------------Actor
\multirow{5}{*}{\begin{sideways}Actor\end{sideways}} & & $\inferrule*[left = ( Receive)]{q\boverto{m?}q'}{(e,q,\sigma)\aoverto{m?}(e,q',\sigma)}$ & 
$\inferrule*[left = (Internal)]{e,\sigma\stoverto{\tau}e',\sigma'}
{(e,q,\sigma)\aoverto{\tau}(e',q,\sigma')}$ \\[1mm]
& \multicolumn{3}{c|}{$\inferrule*[left = (Take)]{q\boverto{m!}q'}{(e,q,\epsilon)\aoverto{m}(e,q',\msgHandlr(m.\name))}$} \\[1mm]
&\multicolumn{3}{c|}{$\inferrule*[left = (EnvSync 1)]{e(\now)<e(\rt)\\e(\now)<t\le e(\rt)}{(e,b,\sigma)\aoverto{t}(e,b,\sigma)}$}\\[1mm] 
&\multicolumn{3}{c|}{$\inferrule*[left = (EnvSync 2)]{
%\include(m,b) \\ \forall m'\cdot (\include(m',b)\Rightarrow m.\ar\le m'.\ar)\\e(\now)<t< m.\ar
{\it isEmpty}(b)
}{(e,b,\epsilon)\aoverto{t}(e,b,\epsilon)}$}\\[1mm]
\cline{2-4}
%---------------Statement
& \multirow{3}{*}{\begin{sideways}Statement\end{sideways}} &
\multicolumn{1}{|c}{$\inferrule*[left = (Cond 1)]{\eval(\expr,e)\\e,\sigma_1\stoverto{\tau}e',\top}{e,{\it if}~ (\expr) ~\sigma_1 ~{\it else}~\sigma_2 \stoverto{\tau}e',\top}$} & $\inferrule*[left = (Seq)]{e,\sigma_1\stoverto{\tau}e,\top}{e,\sigma_1;\sigma_2\stoverto{\tau}e,\sigma_2}$\\[1mm]
& &  \multicolumn{1}{|c}{$\inferrule*[left = (Cond 2)]{\neg\eval(\expr,e)\\e,\sigma_2\stoverto{\tau}e',\top}{e,{\it if}~ (\expr)~ \sigma_1 ~{\it else}~\sigma_2 \stoverto{\tau}e',\top}$} & $\inferrule*[left = (Send)]{}{e,y!\mathfrak{m} \stoverto{(e({\it self}),\mathfrak{m},y)!} e,\top}$\\[1mm]
& & \multicolumn{2}{|c|}{$\inferrule*[left = (Assign)]{}{e,v= \expr \stoverto{\tau} e[v\mapsto \eval(\expr,e)],\top}$} \\[1mm]
& & \multicolumn{2}{|c|}{$\inferrule*[left = (After)]{}{e,y!\mathfrak{m}~{\it after} ~\expr\stoverto{(e({\it self}),\mathfrak{m},y,e(\now)+\eval(\expr,e))!} e,\top}$} \\[2mm]
& & \multicolumn{2}{|c|}{$\inferrule*[left = (Delay)]{}{e,\delay(\expr)\stoverto{\tau} e[\rt\mapsto e(\now)+\eval(\expr,e)],{\it resume}}$} \\[2mm]
& & \multicolumn{2}{|c|}{$\inferrule*[left = (Resume)]{e(\now)==e(\rt)}{e,{\it resume}\stoverto{\tau}e[\rt\mapsto \bot],\top}$}\\
\hline
%-------------Network
{\begin{sideways}Network\end{sideways} & &   $\inferrule*[left = (Transfer)]{b\boverto{m!}b'\\m.\ar==e(\now)}{(e,b)\noverto{m!}(e,b')}$ & $\inferrule*[left = (Receive)]{b\boverto{m?}b'}{(e,b)\noverto{m?}(e,b')}$\\[1mm] 
& \multicolumn{3}{c|}{$\inferrule*[left = (EnvSync 1)]{\include(m,b) \\\forall m'\cdot (\include(m',b)\Rightarrow m.\ar\le m'.\ar)\\e(\now)<t\le m.\ar}{(e,b)\noverto{t}(e,b)}$}\\[1mm]
&\multicolumn{3}{c|}{$\inferrule*[left = (EnvSync 2)]{
%\include(m,b) \\ \forall m'\cdot (\include(m',b)\Rightarrow m.\ar\le m'.\ar)\\e(\now)<t< m.\ar
{\it isEmpty}(b)
}{(e,b)\aoverto{t}(e,b)}$}\\[1mm]\hline
%---------------System
\multirow{3}{*}{\begin{sideways}System\end{sideways}} &  \multicolumn{3}{c|}{$\inferrule*[left = (actorProgress)]{s(x)==(e^\ast,b,\sigma) \\ (e^\ast, b, \sigma)\aoverto{\tau}(e',b',\sigma')}{(e,s,n)\overto{\tau}(e,s[x\mapsto(e',b',\sigma')],n)}$}\\[1mm]
&\multicolumn{3}{c|}{$\inferrule*[left = (comm 1)]{s(x)==(e^\ast,q,\sigma)\\(e^\ast, q, \sigma)\aoverto{m!}(e',q',\sigma')\\ n \noverto{m?}n'}{(e,s,n)\overto{\gamma}(e,s[x\mapsto(e',q',\sigma')],n')}$}\\[1mm]
&\multicolumn{3}{c|}{$\inferrule*[left = (comm 2)]{n\noverto{m!}n'\\m.rcv==x \\ s(x)==(e^\ast,q,\sigma)\\(e^\ast, q, \sigma)\aoverto{m?}(e',q',\sigma')}{(e,s,n)\overto{\gamma}(e,s[x\mapsto(e',q',\sigma')],n')}$}\\
& \multicolumn{3}{c|}{$\inferrule*[left = (EnvProgress)]{\cur==e(\now)\\\exists t\cdot(\forall x\in \ID\cdot (s(x)\aoverto{t}s(x))\wedge n\noverto{t}n)}{(e,s,n)\overto{t}(e[\now\mapsto \cur+t],s,n)}$}\\
\hline
\end{tabular}
\end{table}


As the bag of actors are unbounded, the interface function $\put$ has no limitation and its $?{\it putPolicy}$ is set to true. The interface function $\select$ non-deterministically chooses a message amongst the ones with the minimal arrival time from the bag. In other words, the selection policy is based on the arrival time of messages.


The local state of an actor,  specified by $\langle id, V,\msgHandlr\rangle$, is defined by the triple $(e,b,\sigma)$ where $e$ define the value of state variables of actors and two reserved variables $\now$, denoting the global time, and $\rt$, denoting the resume time of an actor. In other words, $V\cup\{\self,\now,\rt\}=\domain(e)$ such that $e(\self)=id$. The rules \textsc{Receive} and \textsc{Internal} of the generic framework has no assumption, and we only define the assumptions of the rule \textsc{Take} and those of the statements. An actor processes messages based on their arrival time. As this policy is implemented by the select policy of buffer, we set the assumption $?{\it takePolicy}(e,b)$ to $\true$. This setting has two rules for the progress of time. When a rebec executes a delay statements, it should be suspended for the duration specified by the delay statement. Assume that the variable $\rt$ shows the time that the rebec should resume its execution. So, the time can progress from the point view of a rebec as long as it is suspended ($e(\now)<e(\rt)$) and not resumed yet ($e(\now)<t\le e(\rt)$). This is explained by the rule \textsc{EnvSync 1}. The rule \textsc{EnvSync 2} expresses that time can progress as long as the actor has no message to process and is not also busy with processing any message.

This setting inherits all rules for the statements of the core Rebeca, given in Table \ref{Tab::ClassicRules}. We add the semantic rules of the newly added statements. The rule \textsc{After} defines the behavior for the send statement with after and the rule \textsc{Delay} for delay statement. Upon executing a delay statement, the environment variable $\rt$ is advanced with the amount specified in the delay statement. Then, the execution continues by executing the special statement $\it resume$. The rule \textsc{Resume} expresses that this statements can successfully terminates when the global time is equal to the resume time. As a result, the resume time is updated to $\bot$.

The local state of the abstract network entity is defined in terms of the pair $(e,b)$ where $b\in \buffer$ and $e$ contains only one variable $\now$, denoting the global time. The network-level rule \textsc{Receive} is unchanged and only the rules \textsc{Transfer} and \textsc{EnvSync} should be modified. The rule \textsc{Transfer} is defined by specifying the transfer policy. A message is delivered to its receiver when its arrival time is equal to the global time. The rule \textsc{EnvSync 1} explains that time can progress as long as no message can be delivered. In case that the network buffer has no pending message, the time can pass from the point view of the network as expressed by the rule \textsc{EnvSync 2}.

The global state of the system is defined by the triple $(e,s,n)$ where $e$ has only one variable called $\now$ denoting the global time. This setting inherits all the rules of the generic framework. Since no priority is considered for the delivery of messages, the assumption $?{\it priorityPolicy}$ is set to true in the rule \textsc{Comm 2}. The rule \textsc{EnvProgress} expresses that the global time can progress if all the actors and abstract network entity agree with this advancement and can synchronize with this progress. %In this rule $s'$ is defined as $x\in \ID$ if $s(x)\aoverto{t}(e',b',\sigma')$, then $s'(x)=(e',b',\sigma')$.


%-------------------------------------------





\subsection{ABS}

The ABS modeling language \cite{DBLP:conf/fmco/JohnsenHSSS10} is designed for filling the gap between structural modeling languages and implementation-close formalisms. ABS is designed in the object-oriented paradigm to make it easy to use for programmers. It also supports abstractions which are not supported in implementation languages (including functional data types, flexible concurrency and communication constructs, and cooperative scheduling). These abstractions make ABS models more configurable. 

The concurrency model of ABS is based on concurrent objects, asynchronous method calls, and futures. Concurrent objects may be composed into concurrent object groups (COGs). Conceptually, each cog has a dedicated processor and lives in a distributed environment with asynchronous and unordered communication. Considering a cog, at most one process among the objects of the cog is active and the other processes are suspended. Processes in ABS are scheduled by nondeterministic policy, that is controlled by processor release points in a cooperative way. This way, the concurrency level in an ABS model is directly reflected in the number of cogs of the model. All communication is between named objects, typed by interfaces, by means of asynchronous method calls. Calls are asynchronous and the caller is able to decide at runtime when to synchronize with the response of a call.

%Active behavior, triggered by an optional method run, is interleaved with passive behavior, triggered by asynchronous method calls. Thus, an object has a set of processes to be executed, which originate from method activations. Among these, 
%A Creol-like concurrent object model corresponds to an ABS model in which each object has its own cog.
\begin{example}
The code of Figure~\ref{fig:abs:phils} presents the model of the dining philosophers problem in ABS. As shown in the main part (lines 30-39) there are two forks and two philosophers in this model. The initialization of the model takes place by sending the \texttt{behave} message to philosophers in lines 34 and 36. As shown by interface definitions, the \texttt{Philosophers} class has one message handler (i.e. \texttt{behave}, depicted in line 2) and the \texttt{Fork} class has two message handler (i.e. \texttt{grab} and \texttt{grab_second} in lines 5 and 6). Each philosopher takes its first fork by sending a grab message to (line 11), and the second fork is taken by sending grab_second message (line 21).
\end{example}
\begin{figure}[h]
	\centering
	\lstinputlisting[language=ABS]{"./Code/ABS.tex"}
	\caption{A model of dining philosophers in  ABS.}
	\label{fig:abs:phils}
\end{figure}

ABS models can be easily converted into the system abstract syntax model of Section~\ref{sec::AbstractSyntaxModel}. For simplicity, we have assumed that messages in ABS has no parameter. The set of $C$ is defined by the identifier of classes of the model and $\ID$ is defined by the name of instantiated objects from classes of $C$. For each object of class $c$, the class methods are $\msgHandlr$ of the object and class state variables are $\Var$. The name of all methods define the set of $\MName$. The set of $\Stmt$ for defining the body of methods includes:\begin{itemize}
    \item send statement $x!\mathfrak{m}$ that sends a message with the name $\mathfrak{m}\in\MName$ to the object with the identifier $x$.
    \item assignment $v={\it expr}$ that assigns the value of the expression ${\it expr}$ to $v\in\Var$;
    \item sequential composition $s_1;s_2$ that makes the two statements $s_1$ and $s_2$ execute sequentially.
    \item conditional statement ${\it if}~ (\expr) ~s_1 ~{\it else}~s_2 $ that executes the statement $s_1$ and $s_2$ based on the Boolean value of $\expr$,
\end{itemize}
Note that we do not support \emph{future} statement of ABS in this work.

\subsubsection{Structure of Messages and Buffers}
The set of messages in asynchronous message calls of ABS is defined as $\Msg = \ID \times \Var \times \ID$ which is a tuple of the identifier of the sender object, the method name, and the identifier of the receiver object. We use dot notation $m.snd$, $m.rcv$, and $m.name$ to access the sender object identifiers, the receiver object identifiers, and the message handler name of a given message $m$, respectively. %We define the structure of object buffers in ABS as a multiset.


In this setting the structure of objects and network buffers are identical and it is defined as a multiset. A multiset is a map that stores the multiplicity for each distinct member. %For a given multiset $ms$, the multiplicity of an element $e$ is denoted by $\nu(ms, e)$.
%We define the structure of object buffers as a multiset of messages by the two constructors $\empt: \buffer_{\it Actor}$, denoting an empty buffer, and $\frown: \Msg \times \int \times \buffer_{\it Actor}  \rightarrow \buffer_{\it Actor}$, puts a message in the buffer. 
We define the following functions on $\buffer_{\it Actor}$ to manipulate it:
\begin{itemize} 
\item $\insert:\buffer_{\it Actor} \times \Msg\rightarrow \buffer_{\it Actor}$: given a buffer and message, this function inserts the message into the buffer;
\item $contains:\buffer_{\it Actor} \times \Msg \rightarrow boolean$: specifies that if the multiset contains the given message;
\item $\remove:\buffer_{\it Actor} \times \Msg \rightarrow \buffer_{\it Actor} $: it returns a multiset by removing one instance of the given $\Msg$ from the multiset;
\item $\size:\buffer_{\it Actor} \rightarrow \int$: denotes the number of elements in the buffer.
\end{itemize}

We define the same set of functions for $\buffer_{\it Net}$ to manipulate it:
\begin{itemize} 
\item $\insert:\buffer_{\it Net} \times \Msg\rightarrow \buffer_{\it Actor}$: given a buffer and message, this function inserts the message into the buffer;
\item $contains:\buffer_{\it Net} \times \Msg \rightarrow boolean$: specifies that if the multiset contains the given message;
\item $\remove:\buffer_{\it Net} \times \Msg \rightarrow \buffer_{\it Net} $: it returns a multiset by removing one instance of the given $\Msg$ from the multiset;
\item $\size:\buffer_{\it Net} \rightarrow \int$: denotes the number of elements in the buffer.
%\item ${\it cnt}:\buffer_{\it Actor} \times \Msg \rightarrow \int$: denotes the multiplicity of given message in the given multiset .
\end{itemize}
The formal description of multiset structures and their functions are given in Section \ref{sec::abs::core-data}.


%Mapping functions with the same semantics can be defined for $\buffer_{\it Net}$.

%For a given multiset, we use $\oplus$ and $\ominus$ operators to add a message to the buffer and to remove a message from the buffer nondeterministically. The operations of \emph{put} and \emph{select} are defined using $\oplus$ and $\ominus$ as the following. 


\subsubsection{Semantic Rules}
The semantic rules derived from the generic template are depicted in Table~\ref{Tab::ABS-semantic} for ABS.

As the buffers of objects are unbounded, the interface function $\insert$ has no limitation and its $?{\it putPolicy}$ is set to true. Messages can be selected from objects' buffers nondeterministically; so, the only condition for the $Select$ rule is containing a message. The same argument is valid for the buffer of the network.

The local state of an object inform of $\langle id, V, \msgHandlr\rangle$, is defined by the triple $(e,b, \sigma)$ where $e$ defines the value of the state variables of the object, $b$ is its internal buffer and $\sigma$ is the sequence of currently executing statements. Considering the environment $e$ of an object, $V\cup\{\mathit{this}, cog\}=\domain(e)$, $e(\mathit{this})=id$, and $cog:ID\rightarrow Boolean$ is a function that shows is the corresponding cog of an object has a free executing thread.  % which are different that of in the standard semantics rules.
Objects can send messages to themselves by using the reserved word $\it this$. ABS has three actor-level rules \textsc{Receive}, \textsc{Take}, and \textsc{Internal} for showing the behavior of the object (which are the same as the generic framework), and five rules for its statements. The rule \textsc{Send} defines the behavior for the send statement, the rule \textsc{Assign} for assignment, and the rule \textsc{Seq} for the sequential composition. In the rule \textsc{Assign}, $\eval(\expr,e)$ is the value of the expression $\expr$ with respect to the environment $e$. The two rules \textsc{Cond 1} and \textsc{Cond 2} express the behavior of the conditional statement. 

ABS has two network-level rules \textsc{Receive} and \textsc{Transfer} and no rule for modifying the environment as the domain of the environment for the network is empty. The network receives the incoming messages unconditionally and delivers them nondeterministically, so $?{\it transferPolicy(e,b)}$ is set to $\true$ in the rule \textsc{Transfer}.

Considering the system-level rules, the domain of environment for the system only contains ids of cogs. The boolean value which is associated with each cog id in the environment shows that the thread of that cog is assigned to an object or not.  Progress of actors happens in different cases. As shown in \textsc{ActorProgress 1}, a not busy object can start serving one of its received messages if the thread which is associated with its corresponding cog has not been assigned to another object of that cog. This rule results in assigning the thread to that object which is shown by $e[cog(x)\mapsto true]$. \textsc{ActorProgress 2} presents the continuation of executing an object which has an assigned thread from its corresponding cog. Releasing the thread by an object is shown in \textsc{ActorProgress 3} which happens upon executing the last statement of a method body of an object. \textsc{Comm 1} rule for ABS is the same as that of the template framework. Since no priority is considered for the delivery of messages in ABS, the assumption $?{\it priorityPolicy}$ is set to true in the rule \textsc{Comm 2}. There is no rule for updating the environment as only the domain of the system-level environment is not empty and it is updated by its own rules.

%The value which is assigned As for any $e^\ast\in\envT$, $e^\ast\cup e=e$ and $e^\ast\setminus e=e$ when $\domain(e)=\emptyset$, the rules at the system level are simplified as illustrated in Table \ref{Tab::ClassicRules}. Since no priority is considered for the delivery of messages, the assumption $?{\it priorityPolicy}$ is set to true in the rule \textsc{Comm 2}. \fixme{Explain this more!} %As this setting has no time concept, there is no need to define the rule \textsc{EnvSync}.

%-------------------------------
\begin{table}[]
\centering
\caption{The semantic rules of ABS: $q\in \buffer_{\it Actor}$ and $b\in \buffer_{\it Net}$}
\label{Tab::ABS-semantic}
\begin{tabular}{|l|ccc|}
\hline
\multirow{2}{*}{\begin{sideways}Buffer\end{sideways}} 
& \multicolumn{1}{c|} {\begin{sideways}Actor\end{sideways}}& $\inferrule*[left = (Select)]{contains(q, m)}{q\boverto{m!}\remove(q, m)}$ & $\inferrule*[left = (Put)]{q\boverto{m?}\insert(q, m)}{}$\\[1mm] \cline{2-4} 
& \multicolumn{1}{c|} {\begin{sideways}Network\end{sideways}}& $\inferrule*[left = (Select)]{contains(b, m)}{b\boverto{m!}\remove(b, m)}$ & $\inferrule*[left = (Put)]{b\boverto{m?}\insert(b, m)}{}$\\[1mm]
\hline
%------------Actor
\multirow{5}{*}{\begin{sideways}Actor\end{sideways}} & & $\inferrule*[left = ( Receive)]{q\boverto{m?}q'}{(e,q,\sigma)\aoverto{m?}(e,q',\sigma)}$ & 
$\inferrule*[left = (Internal)]{e,\sigma\stoverto{\tau}e',\sigma'}
{(e,q,\sigma)\aoverto{\tau}(e',q,\sigma')}$ \\[1mm]
& \multicolumn{3}{c|}{$\inferrule*[left = (Take)]{q\boverto{m!}q'}{(e,q,\epsilon)\aoverto{m}(e,q',\msgHandlr(m.\name))}$} \\[1mm] \cline{2-4}
%---------------Statement
& \multirow{3}{*}{\begin{sideways}Statement\end{sideways}} &
\multicolumn{1}{|c}{$\inferrule*[left = (Cond 1)]{\eval(\expr,e)\\e,\sigma_1\stoverto{\tau}e',\top}{e,{\it if}~ (\expr) ~\sigma_1 ~{\it else}~\sigma_2 \stoverto{\tau}e',\top}$} & $\inferrule*[left = (Seq)]{e,\sigma_1\stoverto{\tau}e,\top}{e,\sigma_1;\sigma_2\stoverto{\tau}e,\sigma_2}$\\[1mm]
& &  \multicolumn{1}{|c}{$\inferrule*[left = (Cond 2)]{\neg\eval(\expr,e)\\e,\sigma_2\stoverto{\tau}e',\top}{e,{\it if}~ (\expr)~ \sigma_1 ~{\it else}~\sigma_2 \stoverto{\tau}e',\top}$} & $\inferrule*[left = (Send)]{}{e,y!\mathfrak{m} \stoverto{(e({\it this}),\mathfrak{m},y)!} e,\top}$\\[1mm]
& & \multicolumn{2}{|c|}{$\inferrule*[left = (Assign)]{}{e,v= \expr \stoverto{\tau} e[v\mapsto \eval(\expr,e)],\top}$} \\[1mm]
\hline
%-------------Network
\begin{sideways}Network\end{sideways} & & $\inferrule*[left = (Receive)]{b\boverto{m?}b'}{(e,b)\noverto{m?}(e,b')}$   & $\inferrule*[left = (Transfer)]{b\boverto{m!}b'}{(e,b)\noverto{m!}(e,b')}$ \\[1mm] 
\hline
%---------------System
\multirow{5}{*}{\begin{sideways}System\end{sideways}} 
&  \multicolumn{3}{c|}{$\inferrule*[left = (ActorProgress 1)]{s(x)=(e',q,\epsilon) \\ (e', q, \epsilon)\aoverto{m}(e',q',\sigma)\\ e(cog(x)) = \mathit{false}}{(e,s,n)\overto{m}(e[cog(x) \mapsto \mathit{true}],s[x\mapsto(e',q',\sigma)],n)}$}\\[1mm]
&  \multicolumn{3}{c|}{$\inferrule*[left = (ActorProgress 2)]{s(x)=(e^\ast,q,\sigma) \\ (e\cup e^\ast, q, \sigma)\aoverto{\tau}(e',q',\sigma')\\ e(cog(x)) = \mathit{true}}{(e,s,n)\overto{\tau}(e,s[x\mapsto(e'\setminus e,q',\sigma')],n)}$}\\[1mm]
&  \multicolumn{3}{c|}{$\inferrule*[left = (ActorProgress 3)]{s(x)=(e^\ast,q,\sigma) \\ (e\cup e^\ast, q, \sigma)\aoverto{\tau}(e',q',\epsilon)\\ e(cog(x)) = \mathit{true}}{(e,s,n)\overto{\tau}(e[cog(x) \mapsto \mathit{false}],s[x\mapsto(e'\setminus e,q',\epsilon)],n)}$}\\[1mm]
&\multicolumn{3}{c|}{$\inferrule*[left = (comm 1)]{s(x)==(e^\ast,q,\sigma)\\(e^\ast, q, \sigma)\aoverto{m!}(e',q',\sigma')\\ n \noverto{m?}n'}{(e,s,n)\overto{\gamma}(e,s[x\mapsto(e',q',\sigma')],n')}$}\\[1mm]
&\multicolumn{3}{c|}{$\inferrule*[left = (comm 2)]{n\noverto{m!}n'\\m.rcv==x \\ s(x)==(e^\ast,q,\sigma)\\(e^\ast, q, \sigma)\aoverto{m?}(e',q',\sigma')}{(e,s,n)\overto{\gamma}(e,s[x\mapsto(e',q',\sigma')],n')}$}\\
\hline
\end{tabular}
\end{table}


\begin{comment}
\begin{table}[]
\caption{The semantic rules of ABS: $q\in \buffer_{\it Actor}$ and $b\in \buffer_{\it Net}$.}
\label{Tab::ABS-semantic}
\begin{tabular}{|l|cc|}
\hline
\multirow{3}{*}{\begin{sideways}Actor\end{sideways}}  &  $\inferrule*[left = (Act Put)]{q\boverto{m?}\insert(q, m)}{}$ & 
$\inferrule*[left = (Act Select)]{contains(q, m)}{q\boverto{m!}\remove(b, m)}$\\
& $\inferrule*[left = (Assign)]{e,v:=expr \stoverto{\tau}e[v\mapsto \eval(\expr,e)],\top}{}$ &
$\inferrule*[left = (SeqComposition)]{e,\sigma_1\stoverto{\tau}e,\top}{e,\sigma_1;\sigma_2\stoverto{\tau}e,\sigma_2}$ \\
& \multicolumn{2}{|c|}{$\inferrule*[]{\inferrule*[left = (Take)]{q\boverto{m!}q'}{(e,q,\epsilon)\aoverto{m}(e,q',\msgHandlr(m.\name))}}{}$} \\
\hline
\multirow{4}{*}{\begin{sideways}Network\end{sideways}} & \multicolumn{2}{|c|}{$\inferrule*[]{?transferPolicy(e,b,m): true}{}$}\\
& \multicolumn{2}{|c|}{$\inferrule*[left = (receive)]{b\boverto{m?}\insert(b, m)}{}$} \\
\hline
\begin{sideways}System\end{sideways} & \multicolumn{2}{|c|}{$\inferrule*[left = (actorProgress)]{s(x)=(e',q,\epsilon) \\ (e', q, \epsilon)\aoverto{m}(e',q',\sigma)\\ e(cog(x)) = \mathit{false}}{(e,s,n)\overto{m}(e[cog(x) \mapsto \mathit{true}],s[x\mapsto(e',q',\sigma)],n)}$} \\
& \multicolumn{2}{|c|}{$\inferrule*[left = (actorProgress)]{s(x)=(e^\ast,q,\sigma) \\ (e\cup e^\ast, q, \sigma)\aoverto{\tau}(e',q',\sigma')\\ e(cog(x)) = \mathit{true}}{(e,s,n)\overto{\tau}(e,s[x\mapsto(e'\setminus e,q',\sigma')],n)}$} \\
& \multicolumn{2}{|c|}{$\inferrule*[left = (actorProgress)]{s(x)=(e^\ast,q,\sigma) \\ (e\cup e^\ast, q, \sigma)\aoverto{\tau}(e',q',\epsilon)\\ e(cog(x)) = \mathit{true}}{(e,s,n)\overto{\tau}(e[cog(x) \mapsto \mathit{false}],s[x\mapsto(e'\setminus e,q',\epsilon)],n)}$} \\

\hline

\end{tabular}
\end{table}

\end{comment}


\begin{comment}
%Three rules in Fig 8 of the paper
%http://einarj.at.ifi.uio.no/Papers/johnsen10fmco.pdf
%are like our rules for 
%send
%transfer
%receive
%take

%The binding rule is the merge transfer and receive

%Their rules are simple because they have no ordering, in Rebeca we have %...

%How to model Future in ABS using Transparent Actors


%Making explicit all the assumptions on Network and how messages are transferred.
%ABS is abstracting that away.
%LF is what?
%Rebeca making it clear?

%Historically: 
ABS is based on concurrent objects, but they have a layer matched to actors
%LF is based on Hardware/embedded systems DE with ports 

%but they can be both seen as actors.

%In Hybrid Rebeca we differentiate different network transfers
%Gave us the idea of Transparent Actors ...
%ABS has it implicit that there is no order
%Rebeca has an order (partial order)
%LF is completely ordered ... (total order? not really ...)

\subsection{ABS Abstract Network Entity}
%AbsNetwork = multiSet(Msg)
%Msg = Process (ID , message name, param, future)

%\send(message, abs) = abs \cup message

%\transfer (abs) = 
%{
%    message \in abs 
%      (message.receiver). receive(message) 
%}

	\begin{figure}[htbp]
		\centering
		\lstinputlisting[language=Java]%, multicols=2
		{./Fig/ABSAbstractNet.tex}
		\caption{The ABS abstract network entity.}
		\label{fig::ABSabstractNet.tex}
	\end{figure}
	
\subsection{ABS Buffer Entity}
%------------------------------------
%class Buffer {
%    MultiSet<Msg> container;
%    receive(m: Msg) {
%        container.add(m);
%    }
%    take() {
%        return container.nondeterministicPick();
%    }
%}
%class Buffer 
%{
%   container : multiSet(Msg)
   
%   void receive (messsage : Msg)
%   {
%    container.add(message);
%   }
%   message take()
%   {
%        return message \in container
%   }
% }
%--------------------------------


	\begin{figure}[htbp]
		\centering
		\lstinputlisting[language=Java]%, multicols=2
		{./Fig/ABSBuffer.tex}
		\caption{The ABS bufferk entity.}
		\label{fig::ABSBuffer.tex}
	\end{figure}

\end{comment}
\subsection{Lingua Franca}

\newcommand{\ReName}[0]{\mathit{ReName}}
\newcommand{\outport}[0]{\mathit{out}}
\newcommand{\inport}[0]{\mathit{in}}
\newcommand{\action}[0]{\mathit{ac}}
\newcommand{\reaction}[0]{\mathit{Reaction}}


A given Lingua Franca (LF) model, is a collection of reactors and how they are connected together. Reactors are deterministic actors composed out of routines, called reactions, and coordinated under the DE semantics of Ptolemy. Reactions are invoked in response to trigger events, i.e. changing the value of its given parameters. A reaction can be assumed as a message handler, and messages passing among reactors are timestamped and handled using the order of timestamps. The parameters of a reaction is a list of the identifiers of input or clock variables. Note that messages with identical timestamps are logically simultaneous and are handled in a deterministic order. For this case, the order of the definition of reactions determines their order. Reactors can also  share common state, have input and output ports, and actions. Composition of reactors is realized in the main reactor of the model. The connections among them define the flow of messages from an output port of a reactor to an input port of another one. An output port may be connected to multiple input ports, but an input port can only be connected to one output port \cite{DBLP:conf/cyphy/LohstrohRGDCLS19}. An example of a LF model is shown in Listing~\ref{src::LF}. 

\begin{example} 
The model presented in Listing~\ref{src::LF} contains the \texttt{Ramp} reactor which sends messages, yielding a sequence of numbers beginning at 0 with a time intervals given by its given value to the parameter \texttt{p}. The \texttt{Ramp} reactor defines an output port of type \texttt{int}, named \texttt{outValue}, a clock for triggering reactions periodically, and an input of type \texttt{int}, named \texttt{inValue}. This reactor has two reactions, triggered by the clock and the input \texttt{inValue}. The first reaction increments the value of the state variable count and sends its new value to the \texttt{outValue} port. The second reaction sets the value of the counter to the given value. The \texttt{Print} reactor has an input port that triggers its only reaction which prints the value sent to its input port. The composition of reactors is presented in the main reactor, which contains instances of each of reactors and connects their ports. The bodies of reactions are written in some target language; in our example they are written in \texttt{C}. The target code in the figure is delimited by \texttt{\{=} and \texttt{=\}}.
\end{example} 

\begin{lstlisting}[language=LF, multicols=2, caption=Example of a \texttt{Ramp} feeding into a \texttt{Print} reactor (inspired from a LF model presented in \cite{DBLP:conf/dac/LohstrohSGWGSL19}), label=src::LF]
reactor Ramp(p:int(10)) { 
    input inValue: int;
    output outValue: int;
    state int count(0);
    clock c(p);
    
    reaction(c) -> outValue {=
        count ++;
        set(outValue, count); 
    =}
    
    reaction(inValue) {= 
        count = inValue;
    =} 
}
reactor Print { 
    input inValue:int; 
    reaction(inValue) {=
        printf("%d\n", inValue); 
    =}
}
main reactor System {
    a = new Ramp(p=100);
    b = new Print(); 
    a.outValue -> b.inValue;
}
\end{lstlisting}

A LF model is trivially converted into a system abstract syntax model with a minor extension. An abstract syntax of a reactor is defined as the tuple $\langle id, V, \msgHandlr, \inport, \outport\rangle $ where $id\in \ID$ is the reactor identifier, $V \subseteq \Var$ is the set of local variables, $\msgHandlr : \MName \rightarrow \Stmt^*$ define the set of statements an actor should execute in terms of the reactor identifier, $\inport \subseteq \Var$ and $\outport \subseteq \Var$ are input and output ports respectively. Note that $\MName$ contains the identifier of all of the reactions of reactors of a LF model. %\fixme{Reactions do not have name! what shall we do for MName?}  %, $\action\subseteq \Var$ in the set of physical actions, $\reaction : \ReName \rightarrow \Stmt^*$ define the set of statement an actor should execute in terms of the reactor name.
The set of $\Stmt$ for defining the body of reactor includes:\begin{itemize}
    \item set value $set(v, val)$ that sets the value of the output port $v$ to the value $val$.
    \item assignment $v={\it expr}$ that assigns the value of the expression ${\it expr}$ to $v\in\Var$.
    \item sequential composition $s_1;s_2$ that makes the two statements $s_1$ and $s_2$ execute sequentially.
\end{itemize}
Note that as the syntax of LF reactions is target language dependent, we focus on the modeling of reactions containing the above statements not the details of reactions statements.


\subsubsection{Structure of Messages and Buffers}
Messages in LF are tuples of four parts: the reactor identifier, the port name, the sent value, and the message delivery time. So, the set of messages is defined as $\Msg = \ID \times \Var \times \Value \times (\Nat \times \Nat)$. Time in LF is a pair of type $\Nat \times \Nat$. For a given time pair $(st,ms)$, $st$ is called the system time and $ms$ is called the micro-step in the system time. We use dot notation $m.rcv$, $m.prt$, $m.val$, $m.time$ to access the corresponding reactor identifier, the port identifier, the sent value, and the delivery time of a given message $m$.

In this setting the structure of reactors and network buffers are identical and it is defined as a stable priority queue (A priority queue is called stable, if popping elements with the same priority takes place in the same order as they are inserted). We define the following functions on $\buffer_{\it Actor}$ to manipulate it. Here, the priority is defined based on the priority of corresponding reactions of messages. Note that reactions of a reactor has priority which is defined based on the order of the definition of reactions in the LF code.
\begin{itemize} 
\item $\insert:\buffer_{\it Actor} \times \Msg\rightarrow \buffer_{\it Actor}$: given a buffer and message, this function inserts the message into the buffer;
\item $peek:\buffer_{\it Actor} \rightarrow \Msg$: denotes the message that is on the top of the queue;
\item $\remove:\buffer_{\it Actor} \rightarrow \buffer_{\it Actor} $: it returns the message with the highest priority;
\item $\size:\buffer_{\it Actor} \rightarrow \int$: denotes the number of elements in the buffer.
\end{itemize}

We define the same set of functions for $\buffer_{\it Net}$ to manipulate it. Here, the priority is defined based on the delivery time of messages:
\begin{itemize} 
\item $\insert:\buffer_{\it Net} \times \Msg\rightarrow \buffer_{\it Actor}$: given a buffer and message, this function inserts the message into the buffer;
\item $peek:\buffer_{\it Net} \rightarrow \Msg$: denotes the message that is on the top of the queue;
\item $\remove:\buffer_{\it Net} \rightarrow \buffer_{\it Net} $: it returns the message with the highest priority;
\item $\size:\buffer_{\it Net} \rightarrow \int$: denotes the number of elements in the buffer.
\end{itemize}

The formal description of the stable priority queue structures and their functions are given in Section \ref{sec::LF::core-data}.


%Mapping functions with the same semantics can be defined for $\buffer_{\it Net}$.

%For a given multiset, we use $\oplus$ and $\ominus$ operators to add a message to the buffer and to remove a message from the buffer nondeterministically. The operations of \emph{put} and \emph{select} are defined using $\oplus$ and $\ominus$ as the following. 

\subsubsection{Semantic Rules}
Table~\ref{Tab::LF-semantic} illustrates the semantic rules of LF based on the template framework, presented in  Section~\ref{sec::generic}. As each input port of a reactor is connected to only one output port of another reactor, a limited number of messages can be sent to the input ports of each reactor. As a result, a big enough buffer can receive all the incoming messages; so, the interface function $\insert$ has no limitation and its $?{\it putPolicy}$ is set to true. Actors and the network are allowed to pick an element from the top of their priority queue. So, the condition of $Select$ is only having the element at the top of the queue. The same argument is valid for the buffer of the network.

The local state of a reactor inform of $\langle id, V, \msgHandlr, \inport, \outport\rangle$, is defined in terms of the triple $(e,b, \sigma)$ where $e$ defines the value of the state variables of the reactor such that $V\cup\{\mathit{this}\}=\domain(e)$ and $e(\mathit{this})=id$, $b$ is its internal buffer and $\sigma$ is the sequence of currently executing statements. Reactors in LF do not have access to the current time; so, there is no need to define the actor-level rule \textsc{EnvProgress}. This way, LF has three actor-level rules \textsc{Receive}, \textsc{Take}, and \textsc{Internal} for defining the behavior of reactors, and five rules for its statements. The definition of \textsc{Receive}, \textsc{Take}, and \textsc{Internal} rules of LF are the same as that of the template framework. The rule \textsc{SET} defines the behavior for the send messages to the others, the rule \textsc{Assign} for assignment, and the rule \textsc{Seq} for the sequential composition. In the rule \textsc{Assign}, $\eval(\expr,e)$ is the value of the expression $\expr$ with respect to the environment $e$. The two rules \textsc{Cond 1} and \textsc{Cond 2} express the behavior of the conditional statement.

Following the structure of the network in the template framework, the local state of the network is defined in terms of the tuple $(e,b)$ where $e$ defines the value of the state variables of the network and $b$ is its internal buffer. Note that $\{\mathit{now, map, pty}\}=\domain(e)$ and $e(\mathit{now})$ shows the global time of the model. The function $e(\mathit{map})$ is in the form of $\ID \times \Var \rightarrow \ID \times \Var \times \Nat$, contains the topology of a given LF model (including correspondence among input/output ports and their delay times). The $map$ function maps its given reactor id and output port id to a tuple which contains the receiver reactor id, the input port id, and its delay time. The function $e(\mathit{pty})$ is in the form of $\ID \rightarrow \Nat$ which returns the priority of each reactor. As mentioned before, in the case of having two ready to execute reactors, the reactor with the highest priority is executed before the reactor with the lowest priority. 

LF has all of the three network-level rules \textsc{Receive}, \textsc{Transfer}, and \textsc{EnvProgress}. The \textsc{Transfer} rule in LF is the same as that of in the template framework. But, as the network transfers messages with respect to their delivery time, it has a guard to enable transferring for messages which their time is the same as the current time. The \textsc{Receive} rule receives a message containing its output port information and generates its corresponding message which has to be delivered to the receiver reactor using the topology of the model (which includes the receiver id and port, the sent value, and transferring time). The \textsc{Receive} rule is also responsible for taking care of timing issues of the message. If the message is sent without transferring time, the micro-step part of this message has to be set to the next micro-step (i.e. the \textsc{Receive 1} rule). But, if a value is associated with it as the transferring time, the micro-step of the transferring time has to be set to zero (i.e. the \textsc{Receive 2} rule). Finally, rule \textsc{EnvProgress} is defined to illustrate the minimum time progress which enables network to deliver a message.

Considering the system rules, \textsc{ActorProgress}, \textsc{Comm 1}, and \textsc{Comm 2} in LF are the same as that of in the template framework. The rule of \textsc{EnvProgress} in LF is updated to support time progress.

%The value which is assigned As for any $e^\ast\in\envT$, $e^\ast\cup e=e$ and $e^\ast\setminus e=e$ when $\domain(e)=\emptyset$, the rules at the system level are simplified as illustrated in Table \ref{Tab::ClassicRules}. Since no priority is considered for the delivery of messages, the assumption $?{\it priorityPolicy}$ is set to true in the rule \textsc{Comm 2}. \fixme{Explain this more!} %As this setting has no time concept, there is no need to define the rule \textsc{EnvSync}.

%-------------------------------
\begin{table}[]
\centering
\caption{The semantic rules of LF: $q\in \buffer_{\it Actor}$ and $b\in \buffer_{\it Net}$}
\label{Tab::LF-semantic}
\begin{tabular}{|l|ccc|}
\hline
\multirow{2}{*}{\begin{sideways}Buffer\end{sideways}} 
& \multicolumn{1}{c|} {\begin{sideways}Actor\end{sideways}}& $\inferrule*[left = (Select)]{peek(q) = m}{q\boverto{m!}\remove(q)}$ & $\inferrule*[left = (Put)]{q\boverto{m?}\insert(q, m)}{}$\\[1mm] \cline{2-4} 
& \multicolumn{1}{c|} {\begin{sideways}Network\end{sideways}}& $\inferrule*[left = (Select)]{peek(q) = m}{b\boverto{m!}\remove(b)}$ & $\inferrule*[left = (Put)]{b\boverto{m?}\insert(b, m)}{}$\\[1mm]
\hline
%------------Actor
\multirow{5}{*}{\begin{sideways}Actor\end{sideways}} & & $\inferrule*[left = ( Receive)]{q\boverto{m?}q'}{(e,q,\sigma)\aoverto{m?}(e,q',\sigma)}$ & 
$\inferrule*[left = (Internal)]{e,\sigma\stoverto{\tau}e',\sigma'}
{(e,q,\sigma)\aoverto{\tau}(e',q,\sigma')}$ \\[1mm]
& \multicolumn{3}{c|}{$\inferrule*[left = (Take)]{q\boverto{m!}q'}{(e,q,\epsilon)\aoverto{m}(e,q',\msgHandlr(m.\name))}$} \\[1mm] \cline{2-4}
%---------------Statement
& \multirow{3}{*}{\begin{sideways}Statement\end{sideways}} &
\multicolumn{1}{|c}{$\inferrule*[left = (Cond 1)]{\eval(\expr,e)\\e,\sigma_1\stoverto{\tau}e',\top}{e,{\it if}~ (\expr) ~\sigma_1 ~{\it else}~\sigma_2 \stoverto{\tau}e',\top}$} & $\inferrule*[left = (Seq)]{e,\sigma_1\stoverto{\tau}e,\top}{e,\sigma_1;\sigma_2\stoverto{\tau}e,\sigma_2}$\\[1mm]
& &  \multicolumn{1}{|c}{$\inferrule*[left = (Cond 2)]{\neg\eval(\expr,e)\\e,\sigma_2\stoverto{\tau}e',\top}{e,{\it if}~ (\expr)~ \sigma_1 ~{\it else}~\sigma_2 \stoverto{\tau}e',\top}$} & $\inferrule*[left = (Set)]{}{e,(p_o,v) \stoverto{(e(\mathit{this}),p_o,v,(0,0))!}e,\top}$\\[1mm]
& & \multicolumn{2}{|c|}{$\inferrule*[left = (Assign)]{}{e,v= \expr \stoverto{\tau} e[v\mapsto \eval(\expr,e)],\top}$} \\[1mm]
\hline
%-------------Network
\multirow{4}{*}{\begin{sideways}Network\end{sideways}} 
& \multicolumn{3}{|c|}{$\inferrule*[left = (Transfer)]{b\boverto{m!}b'\\ m.time = e(now)}{(e,b)\noverto{m!}(e,b')}$} \\[1mm]
& \multicolumn{3}{|c|}{$\inferrule*[left = (Receive 1)]{m=(a,p,v,(0,0)) \\ (a',p',t')=map(a,p) \\ t' = 0 \\ e(now)=(st,ms) \\ b\overto{(a',p',v,(st, ms + 1))?}b'}{(e,b)\overto{m?}(e,b')}$} \\[1mm]
& \multicolumn{3}{|c|}{$\inferrule*[left = (Receive 2)]{m=(a, p, v, (0,0)) \\ (a', p', t')=map(a, p) \\ t' \neq 0 \\ e(now)=(st,ms) \\ b\overto{(a', p', v, (st + t', 0))?}b'}{(e,b)\overto{m?}(e,b')}$} \\[1mm]
& \multicolumn{3}{|c|}{$\inferrule*[left = (EnvProgress)]{m = peek(b) \\ t=m.time \\ e(now) \neq t} {(e,b)\overto{t}(e[\now\mapsto t],b)}$} \\[1mm]
\hline
%---------------System
\multirow{4}{*}{\begin{sideways}System\end{sideways}} 
&  \multicolumn{3}{c|}{$\inferrule*[left = (ActorProgress)]{s(x)=(e^\ast,b,\sigma) \\ (e\cup e^\ast, b, \sigma)\aoverto{\tau}(e',b',\sigma') \\ \nexists\,y\in \mathit{ID} \cdot s(y)=(e^{\dag},b^{\dag},\sigma^{\dag}) \wedge e(pty(y)) > e(pty(x))\wedge (e\cup e^\dag, b^{\dag}, \sigma^{\dag})\aoverto{\tau}(e'',b'',\sigma'') }{(e,s,n)\overto{\tau}(e,s[x\mapsto(e'\setminus e,b',\sigma')],n)}$}\\[1mm]
&\multicolumn{3}{c|}{$\inferrule*[left = (comm 1)]{s(x)=(e^\ast,q,\sigma)\\(e^\ast, q, \sigma)\aoverto{m!}(e',q',\sigma')\\ n \noverto{m?}n'}{(e,s,n)\overto{\gamma}(e,s[x\mapsto(e',q',\sigma')],n')}$}\\[1mm]
&\multicolumn{3}{c|}{$\inferrule*[left = (comm 2)]{n\noverto{m!}n'\\m.rcv=x \\ s(x)=(e^\ast,q,\sigma)\\(e^\ast, q, \sigma)\aoverto{m?}(e',q',\sigma')}{(e,s,n)\overto{\gamma}(e,s[x\mapsto(e',q',\sigma')],n')}$}\\
&\multicolumn{3}{c|}{$\inferrule*[left = (EnvProgress)]{ n\noverto{t}n'} {(e,s,n)\overto{t}(e[\now\mapsto \cur+t],s,n')}$}\\
\hline
\end{tabular}
\end{table}









\begin{comment}
%----------------------------------------------
% The abstract syntax of the model in LF is the same as the system abstract syntax model.

\subsubsection{Structure of Messages and Buffers}




%In the following we will present an abstract syntax of LF models.

%\begin{defn}[LF Actors Abstract Syntax Model]\label{Def::LFActor}
%An abstract syntax of a LF reactor is defined by the tuple $\langle id, v, \inport, \outport, \action, \reaction\rangle $ where $id\in \ID$ is the reactor identifier, $v\subseteq \Var$ is the set of local variables, $\inport \subseteq \Var$ and $\outport \subseteq \Var$ are input and output ports respectively, $\action\subseteq \Var$ in the set of physical actions, $\reaction : \ReName \rightarrow \Stmt^*$ define the set of statement an actor should execute in terms of the reactor name.
%\end{defn}

% An actor-based system is composed of a set of actors and an abstract network. We define the abstract syntax model of the system: 

% \begin{defn}[System Abstract Syntax Model]\label{Def::absActor}
% An abstract syntax model of an actor-based system $\mathcal{M}$ is defined by the pair $\langle \ID, R,N\rangle $ where $R$ is a set of actor abstract syntax models and $N$ is the abstract model of the network entity.
% \end{defn}

% The messages in this setting consists of three parts: sender identifier, message name, and receiver identifier. So the set of messages is defined as $\Msg = \ID\times\MName\times \ID$. We use dot notation $m.\rcv$ to access the receiver identifier. 

% The set of $\Stmt$ in this setting includes:\begin{itemize}
%     \item send statement $x!\mathfrak{m}$ that sends a message with the name $\mathfrak{m}\in\MName$ to the actor with the identifier $x$.
%     \item assignment $v:={\it expr}$ that assign the value of the expression ${\it expr}$ to $v\in\Var$;
%     \item sequential composition $s_1;s_2$ that makes the two statements $s_1$ and $s_2$ execute sequentially.
% \end{itemize}

Time in LF is a pair of type $\Nat \times \Nat$. For a given time pair $(st,ms)$, $st$ is called the system time and $ms$ is called the micro-step in the system time. Messages in this setting are tuples of four parts: reactor identifier, port name, the sent value, and the message delivery time. So, the set of messages is defined as $\Msg = \ID \times \Var \times \Value \times (\Nat \times \Nat)$. We use dot notation $m.ac$, $m.po$, $m.va$, $m.ti$ to access the corresponding reactor identifier, the port identifier, the sent value, and the delivery time parts of a given message $m$.

As mentioned before, reactions of a reactor has priority which is defined based on the order of definition of reactors in the LF code. So, in the case of having to ready to execute reactors, the reactor with the highest priority is executed before the reactor with the lowest priority. So, we define the structure of reactor buffers in LF as a stable priority queue (A priority queue is called stable, if popping elements with the same priority takes place in the same order as they are inserted). For a given stable priority queue, we use $\oplus$ and $\ominus$ operators to add a message to the queue with an associated priority and to remove the message from the queue that has the highest priority, respectively. The operations of \emph{put} and \emph{select} are defined using $\oplus$ and $\ominus$ as the following. 


\subsubsection{Semantic Rules}
The summary of the semantic rules of LF is presented in Table \ref{Tab::LF-semantic}. Note that the topology of a given LF model (including correspondence among input/output ports and their delay times) is defined by the function $map: \ID \times \Var \rightarrow \ID \times \Var \times \Nat$. This function maps its given reactor id and output port id to a tuple which contains the reactor id and input port id of the destination reactor, together with its delay time.

% Please add the following required packages to your document preamble:
% \usepackage{multirow}
\begin{table}[]
\caption{The semantic rules of Lingua Franca.}
\label{Tab::LF-semantic}
\begin{tabular}{|l|cc|}
\hline
\multirow{3}{*}{\begin{sideways}Actor\end{sideways}}  &  $\inferrule*[left = (Act Put)]{b\boverto{m?}b \oplus m}{}$ & 
$\inferrule*[left = (Act Select)]{m \in b}{b\boverto{m!}b \ominus m}$\\
& $\inferrule*[left = (Set)]{e,(p_o,v) \stoverto{(e(\self),p_o,v,(0,0))!}e,\top}{}$ &
$\inferrule*[left = (Assign)]{e,v:=expr \stoverto{\tau}e[v\mapsto \eval(\expr,e)],\top]}{}$ \\
& $\inferrule*[]{\inferrule*[left = (Take)]{b\boverto{m!}b'}{(e,b,\epsilon)\aoverto{m}(e,b',\msgHandlr(m.\name))}}{}$ & $\inferrule*[left = (SequentialComposition)]{e,\sigma_1\stoverto{\tau}e,\top}{e,\sigma_1;\sigma_2\stoverto{\tau}e,\sigma_2}$ \\
\hline
\multirow{4}{*}{\begin{sideways}Network\end{sideways}} & \multicolumn{2}{|c|}{$\inferrule*[]{?transferPolicy(e,b,m): m.ti = e(now)}{}$}\\
& \multicolumn{2}{|c|}{$\inferrule*[left = (receive)]{m=(a,p,v,(0,0)) \\ (a',p',t')=map(a,p) \\ t' = 0 \\ e(now)=(st,ms) \\ b\overto{(a',p',v,(st, ms + 1))?}b'}{(e,b)\overto{m?}(e,b')}$}
 \\
&  \multicolumn{2}{|c|}{$\inferrule*[left = (receive)]{m=(a, p, v, (0,0)) \\ (a', p', t')=map(a, p) \\ t' \neq 0 \\ e(now)=(st,ms) \\ b\overto{(a', p', v, (st + t', 0))?}b'}{(e,b)\overto{m?}(e,b')}$} \\
& \multicolumn{2}{|c|}{$\inferrule*[left = (EnvSync)]{b\boverto{m!}b \ominus m \\ t=m.ti \\ e(now) \neq t} {(e,b)\overto{t}(e[\now\mapsto t],b)}$} \\
\hline
\begin{sideways}System\end{sideways} & \multicolumn{2}{|c|}{$\inferrule*[left = (EnvProgress)]{ n\noverto{t}n'} {(e,s,n)\overto{t}(e[\now\mapsto \cur+t],s,n')}$} \\
\hline

\end{tabular}
\end{table}

The local state of a reactor inform of $\langle id, V, \msgHandlr, \inport, \outport\rangle$, is defined in terms of the triple $(e,b, \sigma)$ where $e$ defines the value of the state variables of the reactor such that $V\cup\{\self\}=\domain(e)$ and $e(\self)=id$, $b$ is its internal buffer and $\sigma$ is the sequence of currently executing statements.

Actors in LF do not have access to the current time; so, there is no need to define the rule \textsc{EnvSync}. So, LF has three actor-level rules \textsc{Receive}, \textsc{Take}, and \textsc{Internal} for defining the behavior of actor, and three rules for its statements. All of the actor-level rules of LF are the same as that of the generic framework, so they are inherited with no modification. We only define the assumptions $\mathit{?takePolicy}(e,b)$ to always return \texttt{true}. The formal definition of LF basic statements is presented in Table~\ref{Tab::LF-semantic}.



% \begin{comment}
% \begin{itemize}
%     \item this setting has the rule $\receive$ of the generic framework with no modification: 
% $$\inferrule*[left = (receive)]{b\boverto{m?}b'}{(e,b, \sigma)\aoverto{m?}(e, b', \sigma)}$$

%     \item In Lingua Franca, changing the value of an input port results in taking its corresponding message and calling its reactor as shown by the rule $\take$:

% $$\inferrule*[left = (take)]{b\boverto{m!}b' \\ e'=e[m.port\mapsto m.value]}{(e,b,\epsilon)\aoverto{m}(e',b',\msgHandlr(m.name))}$$

% We remark that the second condition of this rule addresses sequential execution of statements, corresponding to the taken message, which is defined as:

% $$\inferrule*[left = (internal)]{e,\sigma \aoverto{\alpha}e', \sigma'}{(e,b,\sigma)\aoverto{\alpha}(e',b,\sigma')}$$

% such that the statement-level rules are
% $$\inferrule*[left = (set)]{e,(p_o,v) \overto{(e(\self),p_o,v,(0,0))}e,\top}{}$$
% $$\inferrule*[left = (assignment)]{e,v:=expr \overto{\tau}e[v\mapsto \eval(\expr,e)],\top]}{}$$

% and $\eval(\expr,e)$ is the value of the expression $\expr$ with respect to the environment $e$. 

% \end{itemize}
% The statement-level rules of LF are:
% $$\inferrule*[left = (Set)]{e,(p_o,v) \stoverto{(e(\self),p_o,v,(0,0))!}e,\top}{}$$
% $$\inferrule*[left = (Assignment)]{e,v:=expr \stoverto{\tau}e[v\mapsto \eval(\expr,e)],\top]}{}$$
% $$\inferrule*[left = (SequentialComposition)]{e,\sigma_1\stoverto{\tau}e,\top}{e,\sigma_1;\sigma_2\stoverto{\tau}e,\sigma_2}$$
% \end{comment}

%\subsubsection{Semantic Rules of Abstract Network}

Following the structure of the network in the general framework, the local state of the abstract network is defined in terms of the tuple $(e,b)$ where $e$ defines the value of the state variables of the network and $b$ is its internal buffer. The \textsc{Transfer} rule in LF is the same as that of in the general framework. But, as the network transfers messages with respect to their delivery time, we modify  $?{\it transferPolicy}$ to enable transferring for messages which their time is the same as the current time. The \textsc{Receive} rule receives a message containing its output port information and generates its corresponding message which has to be delivered to the receiver actor, containing information of the receiver id and port, the sent value, and transferring time using topology of the model. It is also responsible for taking care of timing issues of the message. If the message is sent without transferring time, the micro-step part of this message has to be set to the next micro-step.  But, if a transferring time is associated to binding of the ports of this message, the micro-step of transferring time has to be set to zero. Finally, rule \textsc{EnvSync} is defined to illustrate the minimum time progress which enables network to deliver a message. These three rules are shown in Table~\ref{Tab::LF-semantic}.

Considering the system rules, \textsc{ActorProgress}, \textsc{Comm1}, and \textsc{Comm2} in LF are the same as that of in the general framework. The rule of \textsc{EnvProgress} in LF is updated to support time progress as shown in Table~\ref{Tab::LF-semantic}.
\end{comment}

\begin{comment}

\begin{itemize}
    \item This rules receives a message and generates its corresponding message which has to be delivered to the receiver actor, containing information of the receiver id and port, the sent value, and transferring time using topology of the model. It is also responsible for taking care of timing issues of the message. If the message is sent without transferring time, the micro-step part of this message has to be set to the next micro-step. But, if a transferring time is associated to binding of the ports of this message, the micro-step of transferring time has to be set to zero.
$$\inferrule*[left = (receive)]{m=(a, p, v, (0,0)) \\ (a', p', t')=map(a, p) \\ t' = 0 \\ e(now)=(st,ms) \\ b\overto{(a', p', v, (st, ms + 1))?}b'}{(e,b)\overto{m?}(e,b')}$$
$$\inferrule*[left = (receive)]{m=(a, p, v, (0,0)) \\ (a', p', t')=map(a, p) \\ t' \neq 0 \\ e(now)=(st,ms) \\ b\overto{(a', p', v, (st + t', 0))?}b'}{(e,b)\overto{m?}(e,b')}$$

    \item the rule $\transfer$ only delivers messages to the receiver actors, considering their transferring time.
    $$\inferrule*[left = (transfer)]{m=(a,p,v,t) \in b \\ e(now) = t}{(e,b)\overto{m!}(e,b')}$$

    \item the rule $\mathit{synch}$ updates the value of environment variables with the given set of values.
    $$\inferrule*[left = (synch)]{(e,b)\overto{env}(e \cup env, b)}{}$$

    
As the abstract network entity has no specific behavior, the behavior of its function interfaces are defined in terms of the behavior of its buffer as explicitly shown by the rules in above.

\end{itemize}

\subsubsection{Semantic Rules of Actor System}
The global state of the system is defined by the triple $(e,s,n)$ where $e\in\envT_{\it Sys}$ has only one variable called $\now$ denoting the global time.

\begin{itemize}
    \item this rule is the simple version of the rule in the framework as the global environment is empty.
    $$\inferrule*[left = (Actor Progress)]{s(x)=(e^\ast,b,\sigma) \\ (e^\ast, b, \sigma)\overto{\alpha}(e',b',\sigma') }{(e,s,n)\overto{\alpha}(e,s[x\mapsto(e',b',\sigma')],n)}$$

\item the rule for the interaction from the actor to the network:
$$\inferrule*[left = (Comm_1)]{s(x)=(e^\ast,b,\sigma)\\(e^\ast, b, \sigma)\overto{m!}(e',b',\sigma')\\( e^{\ast\ast},b^{\ast\ast})\overto{m?}(e'',b'')}{(s,(e^{\ast\ast},b^{\ast\ast}))\overto{m\uparrow}(s[x\mapsto(e',b',\sigma')],(e'' ,b''))}$$

\item the rule for the interaction from the network the actor :
$$\inferrule*[left = (Comm_2)]{(e^\ast, b, \sigma)\overto{m?}(e',b',\sigma') \\ ( e^{\ast\ast},b^{\ast\ast})\overto{m!}(e'',b'')}{(s,(e^{\ast\ast},b^{\ast\ast}))\overto{m\downarrow}(s[x\mapsto(e' ,b',\sigma')],(e'' ,b''))}$$


\item As the environment is empty, it has no rule for the progress of the environment. 

\end{itemize}
class Port
{
    name % a unique string, owner::port_name
    priority
}
class LFAbstractNetwork {
    MultiSet<Msg> container;
    Port \rightarrow set(Port) topology;
    %topology(controller::move)=train::move
   
    send(m: Msg) {
        container.add(m);
    }
    %it is assumed that transfer() is called in each micro step
    transfer() {
        container.sort() ; %base on tag timed value
        Set<Msg> enabledMsgs =container.head(); %those having the same timed value as the current model time 
        % coordination 
        queue(Msg) sortedMsgs = topologicalSort(enabledMsgs,topology); %events in each micro step are sorted
        while (!sortedMsgs.isEmpty())
        {
            Msg m = sortedMsgs.head();
            set<Port> ports = topology(m.getSendingPort());
            while (ports!=empty)
            {
                % clone m
                m.setRecievingPort(ports.head());
                Actor actor = getActor(m.getReceiver());
                actor.receive(m);
            }
             sortedMsgs = sortedMsgs.tail();
        }        
    }
    
    %run() {
    % while ....
    %    Msg m = transfer()
    %    
    %}
}
%class Coordinator 
%{
%    micro step
%    macro step 
    % responsible for incrementing the micro/macro step 
    % as the topology is static, we decided to move it to the abstract network 
    % the complication of having topological sort is a consequence of hardware system domains
    
%}

class LFActor {
    % the messages at the port can be read for several times, but can be processed for once. The read can be destructive
    Map<Port, Boolean> isHandled;
    Map<Port, Msg> buffer;

    receive(m: Msg) {
        //If queue is not full
        isHandled.set(m.getReceiverPort(), false); %false means the message has not been handled yet
        buffer.set(m.getReceiverPort(), m);
    }
    % we assumed that the take is non-destructive (the default of LF)
    take() {
        % sort based on the priority of ports
        for(Port port : sort(buffer.getKeys())) {
            % note that when a message is undefined, its corresponding value in isHandled is "true" 
            if(isHandled.get(port) == false) {
                Msg m = buffer.get(port);
                isHandled.set(m.getReceiverPort(), true);
                return m;
            }
        }
    }
    %run() {
    %    Msg m = take()
    %    execute m
    %}
}




the following concepts are mapped:
port to Buffer
topological sort encapsulated in transfer
set value to port = sending
overwritting on port = receiving
taking values from ports = take



Msg : (AID\times MID\times AID \times MID \times Tag) \cup
undefined
% sender ID, sender port, receiver Id, receiver port, Tag (a pair of value)
% AID : actor ID 
% MID : Message name or Port
class AbstractNetwork
{
    container : multiSet(Msg)
    config : Port \rightarrow set(Port)
    send(m: Msg)
    {
        m.setRecievingPort(config(m.sendingPort));
        container.add(m);
    }
    transfer()
    {
        container.sort() ; %base on tag
        set(Msg) tb =container.head();
        queue(Msg) rd = topologicalSort(tb,config);
        while (!rd.isEmpty())
        {
            Msg m = rd.head();
            (m.receivingActor).receive(m);
            rd = rd.tail();
        }
    }
     %run() {
    % while ....
    %    Msg m = transfer()
    %    
    %}
}

class Buffer
{
    b : Port \rightarrow Msg \times Bool
    %reaction and destructive
    %priority over ports
    priority : Port \rightarrow int
    %the lower value, the higher priority
    
    receive(m:Msg)
    {
        b.update(m);
    }
    Msg take()
    {
        for all port p regarding priority
            if !b(p).handled
                Msg temp = b(p).message
                % undefining the message , maybe we can use null
                b.undefine(p)
                return temp 
    }
}

\end{comment}
\subsection{Discussion}
We have summarized how policies are restricted as more aspect added to the Rebeca family in Table \ref{tab:Policy}. 

Decision points: 
Actor puts message in the Buffer.

putPolicy: Overflow, overwrite

selectPolicy: FIFO, EDF, SJF??, pattern matching with no need to a state variable, nonDet

takePolicy: pattern matching using a state variable

transferPolicy: after(t), wiring actors (LF) configuration

priorityPolicy: composition policies, like all messages sent then take




\begin{table}[]
\takcentering
\caption{The policy adjustment for different settings: \xmark denotes a $\true$ assumption while \cmark denotes a more restrictive policy.}
\label{tab:Policy}
\begin{tabular}{|l|l|l|l|l|}
\hline
Level                   & Assumption        & Core  & Timed & Hybrid \\ \hline
\multirow{2}{*}{Buffer} & ${\it putPolicy}$    & \xmark     & \xmark     & \xmark      \\ \cline{2-5} 
                        & ${\it selectPolicy}$  & \cmark & \cmark & \cmark  \\ \hline
Actor                   & ${\it takePolicy}$    & \xmark     & \xmark     & \xmark      \\ \hline
Network                 & ${\it transferPolicy}$  & \xmark     & \cmark & \cmark  \\ \hline
System (Composition rules (Global rules))                  & ${\it priorityPolicy}$  & \xmark     &  \xmark     & \cmark  \\ \hline
\end{tabular}

\end{table}




%\section{Transparent Semantic Model}\label{sec::generic}
The system states are in form of $(s, ns, e)$ where $s$ is the states of actors, $ns$ is the state of the network and $e$ is the set of environment variables.
\subsection{Core Rebeca}
Considering the system state $(s, ns, e)$, we assumed that $e$ is empty in Core Rebeca. Also, we assumed that $ns$ is a function from actor ids to their corresponding queue of messages.
\subsubsection{Actor}
The actor state is a tuple of map of environment variables names to their values, queue content, map of variable names to their values, and the sequence of statements which have to be executed.

\noindent
$\inferrule*[left = (Actor Buffer Take Policy)]{b=\langle m | \delta\rangle}{b \overset{m?}{\longrightarrow}\langle\delta\rangle}$\\
$\inferrule*[left = (Actor Buffer Put Policy)]{b=\langle\delta\rangle}{b \overset{ m!}{\longrightarrow}\langle \delta|m\rangle}$\\
$\inferrule*[left = (Actor Take)]{s(x)=(e, b, v, \epsilon) \\ b \overset{m?}{\longrightarrow} b'}{s \overset{x.m}{\longrightarrow}s[x \mapsto (e, b', v, body(m))]}$\\
$\inferrule*[left = (Actor Receive)]{s(x)=(e, b, v, \langle \sigma \rangle) \\ b \overset{m!}{\longrightarrow} b'}{s \overset{x.m?}{\longrightarrow}s[x \mapsto (e, b', v, \langle \sigma \rangle)]}$\\
$\inferrule*[left = (Actor Send)]{s(x)=(e, b, v, \langle send(m)| \sigma \rangle)}{s \overset{x.m!}{\longrightarrow}s[x \mapsto (e, b, v, \langle \sigma\rangle)]}$\\
$\inferrule*[left = (Actor Assignment Statement)]{\cdots}{\cdots}$\\


\subsubsection{Network}
A network state is a tuple of map of environment variables names to their values and a function from actor ids to the sequence of messages which have to be delivered to them.

\noindent
$\inferrule*[left = (Network Receive)]{ns=(e, nqs) \wedge nqs(x)=\langle q \rangle}{ns \overset{x.m?}{\longrightarrow}(e, nqs[x \mapsto \langle q | m \rangle])}$\\
$\inferrule*[left = (Network Transfer \FG{Policy?})]{ns=(e, nqs) \wedge nqs(x)=\langle m | q \rangle}{ns \overset{x.m!}{\longrightarrow}(e, nqs[x \mapsto \langle q\rangle]}$\\


\subsubsection{System}
$\inferrule*[left = (Sending)]{s \overset{x.m!}{\longrightarrow} s' \\ ns \overset{x.m?}{\longrightarrow} ns'}{(e, s, ns) \overset{x.m\uparrow}{\longrightarrow}(e, s', ns')}$\\
$\inferrule*[left = (Transferring)]{s \overset{x.m?}{\longrightarrow} s' \\ ns \overset{x.m!}{\longrightarrow} ns'}{(e, s, ns) \overset{x.m \downarrow}{\longrightarrow}(e, s', ns')}$\\
$\inferrule*[left = (Internal Transition)]{ s \overset{\Gamma}{\longrightarrow} s'}{(e, s, ns) \overset{\Gamma}{\longrightarrow}(e, s', ns)}$

Note that $\Gamma$ is one of the labels of actors statements execution, including taking a message.


\subsection{Timed Rebeca}
For a given system state $(s, ns, e)$, we assumed that the set of environment variables contains the \emph{now} variable which shows the current time of the actor.
\subsubsection{Actor}
Here, we assumed that each actor has a local variable \emph{now} which shows the current time of the actor. It also contains \emph{res} which shows the time that actor can continue its execution. Both of these variables are initialized to zero. Note that the message container of actors is a multiset (bag) in Timed Rebeca.

\noindent
$\inferrule*[left = (Actor Buffer Take Policy)]{m=(msg, ar, dl) \in b \wedge \forall m'=(msg', ar', dl') \in b \bullet ar \leq ar'}{b \overset{m!}{\longrightarrow}b-m}$\\
$\inferrule*[left = (Actor Buffer Put Policy)]{}{b \overset{m?}{\longrightarrow}b \cup \{m\}}$\\
$\inferrule*[left = (Actor Delay)]{s(x)=(e, b, v, \langle delay(t)| \sigma \rangle) \\ e(now)=t'}{s \overset{delay(t)}{\longrightarrow}s[x \mapsto (e, b, v[res \mapsto t + t'], \langle \sigma \rangle)]}$

For progress of time, there are two different cases. The first one works for an actor which has been executed a delay statement and is waiting for passing the time to execute the next statement.
\newline
$\inferrule*[left = (Actor Time Progress)]{s(x)=(e, b, v, \langle \sigma \rangle) \\ e(now)=t \\ v(res)=t' \\ t < t' \\ t'' \in (t, t']}{s \overset{t'',x}{\longrightarrow}s[x \mapsto (e[now \mapsto t''], b, v, \langle \sigma \rangle)]}$

The second one works for an actor which is free and wants to take a message and execute it but its current time is less than the arrival time of its received messages.
\newline
$\inferrule*[left = (Actor Time Progress)]{s(x)=(e, b, v, \epsilon) \\ e(now)=t \\ \exists\,(msg, ar, dl) \in b \bullet \forall\,(msg', ar', dl') \in b \bullet ar \leq ar' \\ t' \in (t, ar]} {s \overset{t',x}{\longrightarrow}s[x \mapsto (e[now \mapsto t'], b, v, \langle \sigma \rangle)]}$\\

\subsubsection{Network}

A network state is a pair of map from environment variable names to their values and a function from actor ids to the bag of messages which have to be delivered to them.

\noindent
$\inferrule*[left = (Network Receive)]{ns = (e, nbs) \\ nbs(x)=b}{ns \overset{x.m?}{\longrightarrow}(e, nbs[x \mapsto b \cup \{m\}])}$\\
$\inferrule*[left = (Network Transfer)]{ns=(e, nbs) \\ nbs(x)=b \\ m=(msg, ar, dl) \in b \bullet t = ar\FG{t=e(now)}}{ns \overset{x.m!}{\longrightarrow}(e, nbs[x \mapsto b-\{m\}])}$\\
$\inferrule*[left = (Network Time Progress)]{ns=(e, nbs) \\ \exists\,x \in \textit{AID} \wedge \exists (msg, ar, dl) \in nbs(x) \\ \forall\,x' \in \textit{AID} \wedge \forall (msg', ar', dl') \in nbs(x') \wedge ar \leq ar' \\ t' \in (e(now), ar]}{ns \overset{t'}{\longrightarrow}(e[now \mapsto t'], nbs)}$\\

\subsubsection{System Alternative 1}
Rules for \textit{sending}, \textit{transferring}, \textit{internal transitions} are the same as Core Rebeca. Here we define one more rule for having progress in time.

$\inferrule*[left = (Time Progress)]{t \in \mathbb{N} \text{ is the biggest value that } \forall x \in \textit{AID} \bullet s \overset{t,x}{\longrightarrow} s' \\ ns \overset{t}{\longrightarrow} ns' }{(e, s, ns) \overset{t}{\longrightarrow}(e', s, ns)\FG{(e,s',ns'), \mbox{what is $e'=e+t$}}}$

\subsubsection{System Alternative 2}
Rules for \textit{sending}, \textit{transferring}, \textit{internal transitions} are the same as Core Rebeca. Here we define one more rule for having progress in time.

$\inferrule*[left = (Time Progress)]{t \in \mathbb{N} \text{ is the biggest value that } \forall x \in \textit{AID} \bullet  
s(x)=(e, b, v, \Sigma) \\ e(now)=t \\ ((\Sigma \neq \epsilon \wedge (res)=t') \vee (\Sigma = \epsilon \wedge \exists\,(msg, t', dl) \in b \bullet \forall\,(msg', ar', dl') \in b \bullet t' \leq ar') \\ 
ns=(e, nbs) \\ \exists\,x'' \in \textit{AID} \wedge \exists (msg'', t'', dl'') \in nbs(x'') \\ \forall\,x''' \in \textit{AID} \wedge \forall (msg''', ar''', dl''') \in nbs(x'') \wedge t'' \leq ar''' \\
t < t'\wedge t < t'' \\ t''' \in (t, \min\{t', t''\}]}{
(e, s, ns) \overset{t'  ''}{\longrightarrow}(e', s, ns)\FG{\mbox{what about the local e of actors}}}$


%\bibliographystyle{plain}
\bibliography{ref}
\printbibliography{}
\appendix
\section{Formal Specification of Message and Buffer Structures\label{sec::data}}
We make use of equational abstract data types \cite{ADT}. Data is specified
by equational specifications: one can declare data types (so-called sorts) and functions working upon these data types, and
describe the meaning of these functions by equational axioms. 

We use {mCRL2} \cite{Groote} notation to define data types: \emph{sort} declares sort names, \emph{cons} specifies constructor and \emph{map} non-constructor functions, \emph{var} declares variable names, and \emp{eqn} defines non-constructor functions by means of rewrite rules. We assume that the function ${\it if} : Bool \times D \times D$ is defined for all data sorts $D$, which returns the first $D$ parameter if the
boolean parameter equals true, otherwise the second $D$ parameter is returned.
The data sort $Bool$ is used in the conditional operator construct to change the behavior of a process in terms of data
values. This data sort is defined by two constructors $\it true$ and $\it false$ . The conventional operators $\wedge$, $\vee$ and $\neg$ can be defined over it straightforwardly. The data sort $\int$ specifies the natural numbers by the constant $0$ and the unary function $\it succ$. We use $1$, $2$, $\ldots$ for $succ(0)$, $succ(succ(0))$, $\ldots$. The definition of functions $+$, $>$, $\ge$ and $==$ are straightforward.

\subsection{Core Rebeca\label{sec::dataCore}}
We define the structure of actor buffers as an unbounded FIFO queue of messages by the two constructors $\empt: \buffer_{\it Actor}$, denoting an empty buffer, and $\frown: \Msg \times \buffer_{\it Actor}  \rightarrow \buffer_{\it Actor}$, appending a message to the buffer. The definition of $\buffer_{\it Actor}$ and its functions is given in Figure \ref{fig:ActBufferClassic}.
%The behavior of the mappings are formally defined as:\[
%\begin{array}{l}
%\insert(\empt,m) = m\frown \empt\\
%\insert(m'\frown q,m) = m' \frown \insert (q,m) \\\\
%\head(m\frown q) = m \\\\
%\remove(\empt,m) = \empt\\ 
%\remove(m' \frown q,m) = (m==m')\Rightarrow q\wedge (m\neq m')\Rightarrow m'\frown q\\\\
%\size(\empt) = 0\\
%\size(m\frown q) = \size(q)+1
%\end{array}
%\]

\begin{figure}[h]
	\centering
	\lstinputlisting[language=ADT]{"./Code/ActBufferClassic.tex"}
	\caption{Specification of actor buffer in Core Rebeca.}
	\label{fig:ActBufferClassic}
\end{figure}

we define the structure of $\buffer_{\it Net}$ by a set of pairs of actor identifiers and their pending message queue by the two constructors $\empt: \buffer_{\it Net}$, denoting an empty set, and $\triangleright: \ID \times \buffer_{\it Actor} \times \buffer_{\it Net} \rightarrow \buffer_{\it Net}$ appending an actor identifier and its pending messages to the buffer. The definition of $\buffer_{\it Net}$ and its functions is given in Figure \ref{fig:NetBufferClassic}.

%\[
%\begin{array}{l}
%\insert(\empt,m) = (m.\rcv,m\frown\empt)\triangleright \empt\\
%\insert((id,q)\triangleright b,m) = (m.\rcv==id)\Rightarrow (id,\insert(q,m))\triangleright b \, \wedge \\ \hspace*{3.5cm}(m.\rcv\neq id)\Rightarrow ((id,q)\triangleright \insert(b,m)) \\\\
%\remove(\empt,m) = \empt\\ 
%\remove((id,q) \triangleright b,m) = (m.\rcv==id)\Rightarrow (id,\remove(q,m))\triangleright \, b\wedge\\
%\hspace*{4cm}(m.\rcv\neq id)\Rightarrow (id,q)\triangleright \remove(b,m))\\\\
%\end{array}
%\]

\begin{figure}[h]
	\centering
	\lstinputlisting[language=ADT]{"./Code/NetBufferClassic.tex"}
	\caption{Specification of network buffer in Core Rebeca.}
	\label{fig:netBufferClassic}
\end{figure}


\subsection{Timed Rebeca}
In this setting the structure of actor and network buffers are identical and it is defined as an unbounded bag of messages. We define bag by the two constructors $\empt: \buffer$, denoting an empty bag, and $\diamond: \Msg \times \buffer\rightarrow \buffer$, adds a message to the buffer. 

\end{itemize}The behavior of the mappings are formally defined as:\[
\begin{array}{l}
\insert(b,m) = m\diamond b\\\\
\remove(\empt,m) = \empt \\
\remove(m' \diamond b,m) = (m==m')\Rightarrow b\wedge (m\neq m')\Rightarrow m'\diamond\remove(b,m)\\\\
\include(m,\empt) = \false\\
\include(m,m'\diamond b) = (m==m')\Rightarrow \true\wedge (m\neq m')\Rightarrow \include(m,b)
\end{array}
\]

\subsection{ABS}
We define the structure of object buffers as a multiset of messages by the two constructors $\empt: \buffer_{\it Actor}$, denoting an empty buffer, and $\frown: \Msg \times \int \times \buffer_{\it Actor}  \rightarrow \buffer_{\it Actor}$, puts a message in the buffer. We define the following mappings on $\buffer_{\it Actor}$:
\begin{itemize} 
\item $\insert:\buffer_{\it Actor} \times \Msg\rightarrow \buffer_{\it Actor}$: given a buffer and message, this function inserts the message into the buffer;
\item $contains:\buffer_{\it Actor} \times \Msg \rightarrow boolean$: specifies that if the multiset contains the given message;
\item $\remove:\buffer_{\it Actor} \times \Msg \rightarrow \buffer_{\it Actor} $: it returns a multiset by removing one instance of the given $\Msg$ from the multiset;
\item $\size:\buffer_{\it Actor} \rightarrow \int$: denotes the number of elements in the buffer;
\item ${\it cnt}:buffer_{\it Actor} \times \Msg \rightarrow \int$: denotes the multiplicity of given message in the given multiset .
\end{itemize}

The behavior of the mappings are formally defined as:\[
\begin{array}{l}
\insert(\empt,m) = (m,1)\frown \empt\\
\insert((m',n)\frown ms, m) = (m==m') \Rightarrow (m',n+1)\frown ms \wedge (m\neq m')\rightarrow (m',n)\frown \insert(ms,m) \\\\
contains(ms, m) = (\nu(ms, m) > 0) \\\\
\remove(\empt, m) = \empt\\ 
\remove(m' \frown ms, m) = (m==m')\Rightarrow ms\wedge (m\neq m')\Rightarrow m'\frown q\\\\
\size(\empt) = 0\\
\size(m\frown ms) = \size(ms)+1
\end{array}
\]


%--------------------------------------------------------
\subsection{Hybrid Rebeca}
Hybrid Rebeca extends Timed Rebeca with physical behaviors to support modeling of hybrid systems such as embedded or cyber-physical systems. Hybrid Rebeca has two types of classes, reactive and physical. Reactive classes are similar to reactive classes in the Timed Rebeca language where the computational behaviors are defined by message servers. Physical classes in addition to message servers, can also contain different modes, where the continuous behaviors are specified. A physical actor (which is instantiated from a physical class) must always have one active mode. This active mode defines the evolution of continuous variables over time in terms of ordinary differential equations (ODE). By changing the active mode of a physical actor, it is possible to change the continuous behavior of the actor.  The modes of physical classes are similar to the concept of locations in hybrid automata. Rebecs can communicate via wire with a zero delay or CAN bus with different delays. The CAN network transfers messages based on their arrivals and uses a priority policy for the messages that have arrived simultaneously. The delays and priorities are specified in a CAN block in terms of the sender and the message name.

\begin{figure}[h]
	\centering
	\lstinputlisting[language=HRebeca, multicols=2]{"./Code/hybrid.tex"}
	\caption{A model of a heater with a sensor specified in Hybrid Rebeca.}
	\label{fig:sensor}
\end{figure}

\begin{example}
We extend the given model of Figure \ref{fig:Controller} with a model of a heater with a sensor as shown in Figure \ref{fig:sensor}. The new main block connects a monitor to the heater. The heater has two modes $\it On$ and $\it Off$: in the $\it On$ mode, the temperature is increased with the rate of $1$. It sends a $\it check$ message to the monitor when the temperature is above $22$ before switching to the mode $\it Off$. Heater communicates via a CAN bus with the monitor while the monitor is connected by wire to the alarm. The specification of CAN network defines priorities and delays for messages in lines 45-53. 
\end{example}

Each physical rebec has a set of real variables with a set of modes. To specify the behavior of a physical class, two statements have been added: 
\begin{itemize}
    \item mode block ${\it mode}~ {\it name}~ {\it inv}({\it expr}){ode}~{\it guard}({\it expr}){stmt}$: a physical rebec can stay in mode as long as its invariant expressions holds, and the state variables of rebec updates according to the given $\it ode$. A solution of an $\it ode$ is called a flow. The statements $\it stmt$, called trigger of a mode, can be executed when the given guard expression holds.  
    \item set mode statement $x.{\it setmode}(M)$ that sets the active mode of rebec $x$ to $M$.
\end{itemize}
%The active mode of an actor can be changed by using the set mode statement $x.{\it setmode}(M)$ that sets the active mode of rebec $x$ to $M$. 
We extend the actor abstract syntax model given in Definition \ref{Def::absActor} with an extra item ${\it Mode}$ for specifying the mode of physical rebecs. 

\begin{defn}[Extended Actor Abstract Syntax Model]\label{Def::absPhyActor}
An abstract syntax model of a physical actor instance is defined by the vector $\langle id, V,\msgHandlr,\aqu, {\it Mode}\rangle $ where $id\in \ID$ is the actor identifier, $v\subseteq \Var$ is the set of local variables, $\msgHandlr : \MName \rightarrow \Stmt^*$ define the set of statement an actor should execute in terms of the message name, $\aqu\subseteq \ID$ is the set of acquaintances that an actor can communicate with, and $\Mode\subseteq \MoName\rightarrow \Expr\times \ODE \times \Expr \times \Stmt$ is its set of model.
\end{defn}

Each mode is specified in terms of four items called ${\it inv}$, ${\it flow}$, ${\it guard}$, and ${\it trigger}$. Given the name of a mode $\mathfrak{m}$, we assume we can access these for items by $\inv(\Mode(\mathfrak{m}))$, $\flow(\Mode(\mathfrak{m}))$, $\guard(\Mode(\mathfrak{m}))$, and $\trigger(\Mode(\mathfrak{m}))$, respectively. For a flow $\mathcal{F}$, We show its solution by the notation $f\in\mathcal{F}$, where $\frac{df}{dt}=\mathcal{F}$. 

\subsubsection{Semantic Rules}
The structure of messages and buffers in this setting are similar to the Timed setting. This setting almost has the same rules as the timed setting. In the following, we explain the new semantic rules and the ones that should be changed in comparison with the timed setting.  


\begin{table}[]
\centering
\caption{The new and modified semantic rules of the Hybrid Rebeca.}
\label{Tab::HybridRules}
\begin{tabular}{|cc|}
\hline
%\multicolumn{2}{|c|}{$\inferrule*[left = (Select)]{\include(m,b)\\\forall m'\cdot (\include(m',b)\Rightarrow m.\ar\le m'.\ar)}{b\boverto{m!}\remove(b,m) }$}\\[1mm]
%$\inferrule*[left = (put)]{}{b\boverto{m?}\insert(b,m) }$ & $\inferrule*[left = (Take)]{b\boverto{m!}b'}{(e,b,\epsilon)\aoverto{m?}(e,b',\body(m))}$\\[1mm]
%$\inferrule*[left = (Act Receive)]{q\boverto{m?}q'}{(e,q,\sigma)\aoverto{m?}(e,q',\sigma)}$ & %$\inferrule*[left = (Internal)]{e,\sigma\aoverto{\tau}e',\sigma'}{(e,q,\sigma)\aoverto{\tau}(e',q,\sigma')}$ \\[1mm]
%\multicolumn{2}{|c|}{$\inferrule*[left = (Act EnvSync 1)]{e(\now)<e(\rt)\\e(\now)<t\le e(\rt)}{(e,b,\sigma)\aoverto{t}(e,b,\sigma)}$}\\[2mm] 
%\multicolumn{2}{|c|}{$\inferrule*[left = (Act EnvSync 2)]{\include(m,b) \\ \forall m'\cdot (\include(m',b)\Rightarrow m.\ar\le m'.\ar)\\e(\now)<t< m.\ar}{(e,b,\epsilon)\aoverto{t}(e,b,\sigma)}$}\\[2mm]
\multicolumn{2}{|c|}{$\inferrule*[left = (Act EnvSync 3)]{b==\empt\\f\in \flow(\Mode(e({\mode})))\\f(0)==e\\f(t)==e'\\ \forall 0\le t'\le t\cdot(f(t')\models \inv(\Mode(e({\mode}))))}{(e,b,\epsilon)\aoverto{t}(e',b,\sigma)}$}\\[2mm]
\multicolumn{2}{|c|}{$\inferrule*[left = (Setmode)]{}{e,{\it setmode}(\mathfrak{m})\stoverto{\tau} e[\mode\mapsto\mathfrak{m}],\top} $}\\[2mm]
\multicolumn{2}{|c|}{$\inferrule*[left = (EndMode)]{e\models \guard(\Mode(e({\mode}))) }{(e,b,\epsilon)\aoverto{\tau}(e[\mode\mapsto {\it none}],b,\trigger(\Mode(e({\mode}))))}$}\\[3mm]
%\multicolumn{2}{|c|}{$\inferrule*[left = (Net Receive)]{b\boverto{m?}b'}{(e,b)\noverto{m?}(e,b')}$} \\[1mm]
\multicolumn{2}{|c|}{$\inferrule*[left = (Transfer)]{b\boverto{m_1!}b'\\m_1.\ar+e(\delay)(m_1.\sndr,m_1.\name)==e(\now)\\
\nexists m_2\cdot(b\boverto{m_2} \, \wedge\, e(\priority)(m_1.\sndr,m_1.\name)>e(\priority)(m_2.\sndr,m_2.\name) \wedge\\ m_2.\ar+e(\delay)(m_2.\sndr,m_2.\name)\le m_1.\ar+e(\delay)(m_1.\sndr,m_1.\name))\\m.\rcv==m_1.\rcv\\m.\name==m_1.\name \\m.\sndr==m_1.\sndr\\m.\ar==e(\now)+e(\delay)(m_1.\sndr,m_1.\name)}{(e,b)\noverto{m!}(e,b')}$}\\[3mm]
%\multicolumn{2}{|c|}{$\inferrule*[left = (Net EnvSync)]{\include(m,b) \\\forall m'\cdot (\include(m',b)\Rightarrow m.\ar\le m'.\ar)\\e(\now)<t\le m.\ar}{(e,b)\noverto{t}(e,b)}$}\\[1mm]
%\multicolumn{2}{|c|}{$\inferrule*[left = (actorProgress)]{s(x)==(e^\ast,b,\sigma) \\ (e^\ast, b, \sigma)\aoverto{\tau}(e',b',\sigma')}{(e,s,n)\overto{\tau}(e,s[x\mapsto(e',b',\sigma')],n)}$}\\[1mm]
%\multicolumn{2}{|c|}{$\inferrule*[left = (comm 1)]{s(x)==(e^\ast,q,\sigma)\\(e^\ast, q, \sigma)\aoverto{m!}(e',q',\sigma')\\ n \noverto{m?}n'}{(e,s,n)\overto{\gamma}(e,s[x\mapsto(e',q',\sigma')],n')}$}\\[1mm]
\multicolumn{2}{|c|}{$\inferrule*[left = (comm 2)]{n\noverto{m!}n'\\m.rcv==x \\ s(x)==(e^\ast,q,\sigma)\\(e^\ast, q, \sigma)\aoverto{m?}(e',q',\sigma')\\\nexists y\in\ID\cdot(s(y)\aoverto{\tau})}{(e,s,n)\overto{\gamma}(e,s[x\mapsto(e',q',\sigma')],n')}$}\\[1mm]
%\multicolumn{2}{|c|}{$\inferrule*[left = (EnvProgress)]{\cur==e(\now)\\\exists t\cdot(\forall x\in \ID\cdot (s(x)\aoverto{t}s(x))\wedge n\noverto{t}n)}{(e,s,n)\overto{t}(e[\now\mapsto \cur+t],s,n)}$}\\[1mm]

\hline
\end{tabular}
\end{table}

Similar to Timed setting, the local state of an actor is defined in terms of the pair $(e,b,\sigma)$. % where $e\in\envT_{\it Actor}$. We partition $\envT_{\it Actor}$ into the two disjoint environments $\envT_{\it Actor_s}$ and $\envT_{\it Actor_p}$ for software and physical actors, respectively. 
For a software actor, specified by $\langle id, V,\msgHandlr,\aqu\rangle$, %is defined in terms of the triple $(e_s,b,\sigma)$ where 
%its environment $e_s\in\envT_{\it Actor_s}$ is similar to the rebecs of Timed setting, i.e., 
it holds that $V\cup\{\self,\now,\rt\}=\domain(e)$. As physical rebecs has no delay statement, their environment does not contain the variable $\rt$, instead it has the variable $\mode$ indicating the active mode of the rebec. %So for the environment $e_p\in \envT_{\it Actor_p}$ of 
For a physical actor, specified by $\langle id, V,\msgHandlr,\aqu, {\it Mode}\rangle$, %is defined in terms of the triple $(e_p,b,\sigma)$ where
it holds that $V\cup\{\self,\now,\mode\}=\domain(e)$. Reactive rebecs have the same behavior to the ones in the timed setting and all the actor-level rules are valid for this setting. Physical rebecs have a similar behavior to software rebecs on processing their messages. So the rules \textsc{Receive}, \textsc{Internal}, and \textsc{Take} of the timed setting also work for the hybrid setting. We only define the rules for their continuous behaviors. The real-valued variables of a physical rebec update as time passes. So we define the semantic rules \textsc{EnvSync} for the physical rebecs to indicate when they agree with the advancement of time. The rule \textsc{EnvSync 3} expresses that time can passes as long as the invariant of the mode holds, and the variables are updated accordingly.

We define the semantic rule of the newly added statements. The rule \textsc{Setmode} changes the active mode of a rebec. The rule \textsc{EndMode} expresses when the trigger statements of a mode block can be executed. When the guard of the mode holds, the mode is terminated by setting its value to the special mode $\it none$ in which the value of variables are freezed as time passes.%The rule ${\it envSync}$ explains that the time can progress as long as the invariant of the mode is satisfied by the value of variables. This rule extends the actor-level rules for defining the behavior of the actor upon its active mode termination. As explained by the rule, {\it trigger} of the active mode are executed. 


The network environment has two additional variables $\delay$ and $\priority$ to maintain the CAN specification as a mapping. The variable $\delay:\ID\times\MName\rightarrow \Nat$, is itself a mapping that given the identifier of a rebec and communicated message name, it returns a delay value. Similarly variable $\priority:\ID\times\MName\rightarrow \Nat$ , is itself a mapping that given the identifier of a rebec and communicated message name, it returns a priority value. The lower the value, the higher the priority. The network-level rules \textsc{Receive} and \textsc{envSync}  are the same as the timed setting. As the network transfers messages with respect to their priority, we modify the rules \textsc{Transfer} by setting its appropriate $?{\it transferPolicy}$.  This policy selects the message with the highest priority that its delivery time has been arrived.


The system-level rules are the same as the timed setting except for the rule \textsc{Comm 2}. As the network should transfer the message with the highest priority that should be delivered at the same time, the network should first receive all the messages and then transfer. We define the priority assumption such that the network can transfer when no rebec can have progress.  

\end{document}
